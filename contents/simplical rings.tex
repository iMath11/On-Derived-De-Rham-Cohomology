\section{Simplicial Rings}
\subsection{Differential Graded Algebras}

\begin{definition}[Differential graded algebra]\label{def:dga}
Let $A$ be a commutative unital ring.  
A \emph{differential graded algebra} (abbreviated \emph{dga}) over $A$ is a cochain complex
\[
(C^\ast, d) = \bigl(\cdots \to C^{n-1} \xrightarrow{d_{n-1}} C^n \xrightarrow{d_n} C^{n+1} \to \cdots\bigr)
\]
of $A$–modules together with a morphism of cochain complexes
\[
\mu : C^\ast \otimes_A C^\ast \longrightarrow C^\ast,
\]
called the \emph{multiplication}, which is \emph{unital} and \emph{associative} in the obvious sense.
\end{definition}

\begin{definition}[Graded commutative dga]
A differential graded algebra $C^\ast$ over $A$ is said to be \emph{(graded) commutative} if it is endowed with a morphism of cochain complexes
\[
\tau : C^\ast \otimes_A C^\ast \longrightarrow C^\ast \otimes_A C^\ast, 
\qquad
x \otimes y \longmapsto (-1)^{\deg x\, \deg y} \, y \otimes x,
\]
where $\deg x = n$ whenever $x \in C^n$, such that the diagram
\[
\begin{tikzcd}
C^\ast \otimes_A C^\ast \ar[r, "\tau"] \ar[dr, "\mu"'] & C^\ast \otimes_A C^\ast \ar[d, "\mu"] \\
& C^\ast
\end{tikzcd}
\]
commutes.  In that case, we say that $\mu$ is \emph{graded commutative}.
The category of such objects is denoted by $\mathbf{cdga}_A$.
\end{definition}

\begin{remark}
Equivalently, a dga can be described more concretely as a graded $A$–module
\[
C^\ast = \bigoplus_{n \ge 0} C^n
\]
equipped with
\begin{enumerate}
  \item an associative unital multiplication
  \[
  \cdot : C^n \times C^m \longrightarrow C^{n+m},
  \]
  making $\bigoplus_{i \ge 0} C^i$ a graded $A$–algebra; and
  \item a differential $d : C^n \to C^{n+1}$ satisfying $d^2 = 0$
        and the \emph{Leibniz rule}
        \[
        d(x \cdot y) = d(x)\cdot y + (-1)^{n}\, x \cdot d(y),
        \qquad \text{for } x \in C^n,\, y \in C^m.
        \]
\end{enumerate}
The unital and associative property simply means that $C^\ast$ is a graded $A$–algebra, and the differential is compatible with the multiplication via the Leibniz identity above.
\end{remark}

\textbf{Motivation and Origin of Differential Graded Algebras}

The concept of a \emph{differential graded algebra} (dga) emerged as a natural unification
of two fundamental ideas in mathematics:
\begin{itemize}
  \item the \emph{algebraic structure} of multiplication, coming from rings or algebras; and
  \item the \emph{homological structure} of complexes, governed by a differential operator $d$ with $d^2=0$.
\end{itemize}

Historically, the notion appeared first in the work of Henri Cartan and Samuel Eilenberg
(1950s) as part of \emph{homological algebra}. Their goal was to formalize the algebraic
structures underlying cohomology theories.
The ``graded'' part allows us to keep track of degrees of cochains,
while the ``differential'' captures the boundary or coboundary operator from topology.

A fundamental example is the \emph{de Rham complex} of a smooth manifold $M$:
\[
(\Omega^\ast(M), d),
\]
where $\Omega^n(M)$ is the space of differential $n$–forms and
$d: \Omega^n(M) \to \Omega^{n+1}(M)$ is the exterior derivative.
The wedge product $\wedge$ makes it into a graded-commutative algebra,
and $d$ satisfies the Leibniz rule
\[
d(\alpha \wedge \beta)
  = d\alpha \wedge \beta + (-1)^{\deg \alpha} \alpha \wedge d\beta,
\]
so $(\Omega^\ast(M), d, \wedge)$ is a commutative differential graded algebra.

Later, in derived algebraic geometry (Illusie, 1970s), the notion of a dga became
essential to define the \emph{cotangent complex} and the
\emph{derived de~Rham complex}, replacing the smooth differential forms of geometry
by algebraic analogues over arbitrary commutative rings.
In modern language, dg–algebras are basic models for
\emph{chain-level algebraic structures} in homotopical algebra and derived geometry.

\textbf{Intuitive Motivation}

A differential graded algebra encodes both algebra and homology in a single object:
\begin{itemize}
  \item the \textbf{grading} $C = \bigoplus_n C^n$ organizes data by degree;
  \item the \textbf{differential} $d$ measures how elements of one degree
        ``fail to be closed'' and leads to cohomology groups $H^n(C)$;
  \item the \textbf{product} $\mu$ interacts with $d$ by the Leibniz rule,
        allowing multiplicative structures to descend to cohomology.
\end{itemize}

Thus a cdga is a bridge between linear homological data (complexes)
and nonlinear multiplicative phenomena (rings).
For example, the cup product in singular cohomology can be realized
at the level of a cdga model.

\textbf{Examples}

\begin{example}[De~Rham complex]
Let $A = \mathbb{R}$ and let $C^\ast = \Omega^\ast(M)$ be the graded algebra of differential
forms on a smooth manifold $M$ with exterior derivative $d$.
Then $(\Omega^\ast(M), d, \wedge)$ is a commutative dga over $\mathbb{R}$.
\end{example}

\begin{example}[Singular cochains]
For a topological space $X$, the singular cochain complex
$C^\ast(X; A) = \Hom_{\mathbb{Z}}(C_\ast(X), A)$ with the cup product
$\smile$ and the coboundary operator $d$ is a (graded) commutative dga over $A$.
Its cohomology is the singular cohomology ring $H^\ast(X; A)$.
\end{example}

\begin{example}[Trivial dga]
Any graded algebra $C^\ast$ concentrated in degree~$0$
(e.g.~a usual commutative ring $A$ with $C^0=A$, $C^i=0$ for $i>0$)
is a dga with $d=0$.  Its cohomology is itself.
\end{example}

\begin{example}[Koszul complex]
Let $A$ be a commutative ring and $a_1,\dots,a_n\in A$.
The Koszul complex $K^\ast(A; a_1,\dots,a_n)$
is the exterior algebra $\bigwedge_A(e_1,\dots,e_n)$
with differential
\[
d(e_i) = a_i, \qquad d(e_i \wedge e_j) = a_i e_j - a_j e_i, \ \text{etc.}
\]
It is a nontrivial but fundamental example of a commutative dga.
\end{example}

\begin{example}[Endomorphism algebra]
Given any complex of $A$–modules $(M^\ast,d)$, the graded module
$\Hom_A(M^\ast,M^\ast)$ carries a natural dga structure:
the composition of maps gives multiplication, and the differential is
\[
d(f) = d_M \circ f - (-1)^{\deg f} f \circ d_M.
\]
This is typically \emph{not} commutative.
\end{example}

\textbf{Counterexamples and Noncommutative Cases}

\begin{example}[Noncommutative dga]
Let $A$ be any (possibly noncommutative) algebra and set
$C^\ast = A[t, dt]$ where $t$ has degree~$0$ and $dt$ degree~$1$,
with $d(t)=dt$ and $d(dt)=0$.
Multiplication is the usual polynomial one.
Then $(C^\ast, d)$ is a dga but \emph{not} commutative since $t\,dt \ne (-1)^1 dt\,t$.
\end{example}

\begin{example}[Failure of the Leibniz rule]
If one defines a differential $d$ on a graded algebra $C^\ast$
that does \emph{not} satisfy
\[
d(xy)=d(x)y+(-1)^{\deg x}x\,d(y),
\]
then even if $d^2=0$ the structure is no longer a dga.
For instance, if $C^\ast$ is the graded algebra
$\mathbb{R}[x]$ with $\deg x=1$ and $d(x)=1$, then $d^2=0$
but $d(x^2)=2x\ne d(x)x - x d(x)=0$, so the Leibniz rule fails.
\end{example}

\begin{example}[Non-graded commutativity]
Consider matrices $C^0=M_2(\mathbb{R})$, $C^1=M_2(\mathbb{R})$
with $d$ given by $d(A)=BA-AB$ for some fixed matrix $B$.
Then $d^2=0$ but multiplication by matrix product
is noncommutative, so $(C^\ast,d)$ is a dga but not a cdga.
\end{example}

\textbf{Conceptual Summary}

Differential graded algebras are central objects in homological algebra,
rational homotopy theory, and derived algebraic geometry.  
They serve as algebraic models of spaces (via the de~Rham or cochain cdga),
and as tools to encode deformation problems (via Maurer--Cartan equations in dg–Lie algebras).
The ``graded'' structure captures the hierarchy of forms or operations,
and the ``differential'' connects adjacent layers, encoding boundary information.
Commutative dg–algebras (cdgas) are the algebraic heart of derived geometry.

\begin{proposition}\label{prop:cohomology-graded-algebra}
Given a differential graded algebra $(C^\ast, d)$ over a commutative ring~$A$,
its multiplication induces a natural structure of graded $A$–algebra on the
cohomology groups
\[
H^\ast(C^\ast) := \bigoplus_{i \in \mathbb{Z}} H^i(C^\ast),
\qquad
H^i(C^\ast) = \frac{\ker(d : C^i \to C^{i+1})}{\operatorname{im}(d : C^{i-1} \to C^i)}.
\]
\end{proposition}

\begin{proof}[Sketch of proof]
Let $u,v \in C^\ast$ be homogeneous elements.

\begin{enumerate}
  \item[(1)] If both $u$ and $v$ are \emph{cocycles}, i.e.\ $du = dv = 0$, then by the
  Leibniz rule for a dga,
  \[
  d(u \cdot v) = (du)\,v + (-1)^{\deg u}\,u\,(dv) = 0,
  \]
  so $u \cdot v$ is also a cocycle.

  \item[(2)] Suppose $u,u'$ are cocycles differing by a coboundary:
  $u' = u + d\xi$ for some $\xi \in C^{\deg u - 1}$.
  Let $w$ be any cocycle ($dw=0$).
  Then
  \[
  u'w - uw = (u + d\xi)w - uw = (d\xi)w
     = d(\xi w) - (-1)^{\deg \xi} \xi\, d(w)
     = d(\xi w),
  \]
  since $d(w)=0$. Hence $u'w$ and $uw$ differ by a coboundary.
  The same argument applies to $wu$ and $wv$.
\end{enumerate}

These two facts ensure that the product
\[
[u]\,[v] := [u \cdot v]
\]
is \emph{well-defined} on cohomology, i.e.\ it depends only on the cohomology
classes of $u$ and $v$, not on their chosen representatives.
The associativity and unital properties are inherited directly from those of $C^\ast$.

\medskip
Alternatively, one may proceed functorially:
the multiplication map of complexes
\[
\mu : C^\ast \otimes_A C^\ast \longrightarrow C^\ast
\]
induces a graded $A$–linear map on cohomology
\[
H^\ast(C^\ast \otimes_A C^\ast) \xrightarrow{H^\ast(\mu)} H^\ast(C^\ast),
\]
and by the canonical Künneth morphism
\[
H^\ast(C^\ast) \otimes_A H^\ast(C^\ast)
\;\longrightarrow\;
H^\ast(C^\ast \otimes_A C^\ast),
\quad [u] \otimes [v] \longmapsto [u \otimes v],
\]
one obtains the induced multiplication
$[u][v] = [u \cdot v]$ on $H^\ast(C^\ast)$.
\end{proof}

\begin{remark}
Commutative differential graded algebras (cdga's) form a central class of objects in
modern arithmetic geometry and derived algebraic geometry.
Their cohomology rings inherit a natural graded–commutative algebra structure,
but the category of cdga's is difficult to handle within ordinary homotopy theory.
This motivates the development of higher and derived analogues such as
$E_\infty$–algebras, which formalize ``commutativity up to coherent homotopy.''
\end{remark}

\textbf{Motivation and Interpretation}

The purpose of Proposition~\ref{prop:cohomology-graded-algebra} is to show that
the multiplicative structure of a differential graded algebra $(C^\ast,d)$
descends to its cohomology.
Intuitively, cohomology measures the ``closed'' elements up to ``exact'' ones,
and Proposition~\ref{prop:cohomology-graded-algebra} ensures that this quotient
inherits the same algebraic behavior.

In topological or geometric terms, this explains why
the cohomology of a manifold carries a natural ring structure:
the product of closed differential forms is again closed,
and changing a form by an exact one does not alter its class.
This phenomenon generalizes to every dga and formalizes the ``ring of invariants''
that remains after quotienting by the differential.

\textbf{Comments and Remarks}

\begin{remark}[Functorial nature]
The construction is functorial:
a morphism of dg–algebras $f : C^\ast \to D^\ast$ induces a morphism
of graded algebras
\[
H^\ast(f) : H^\ast(C^\ast) \longrightarrow H^\ast(D^\ast).
\]
Thus, cohomology becomes a functor from the category of dg–algebras
to the category of graded algebras.
\end{remark}

\begin{remark}[Graded commutativity]
If $C^\ast$ is a \emph{commutative} dga (cdga),
then its cohomology algebra is \emph{graded commutative}:
for homogeneous classes $[u] \in H^p(C^\ast)$, $[v] \in H^q(C^\ast)$,
\[
[u][v] = (-1)^{pq} [v][u].
\]
This follows from the graded commutativity of multiplication in $C^\ast$ itself.
\end{remark}

\begin{remark}[Homotopy invariance]
Since quasi-isomorphic dga’s have isomorphic cohomology rings,
$H^\ast(C^\ast)$ is an invariant of the quasi-isomorphism class of $C^\ast$.
This property is fundamental in derived algebraic geometry, where
cohomology represents the ``classical shadow'' of a derived object.
\end{remark}

\textbf{Examples}

\begin{example}[De~Rham cohomology]
Let $C^\ast = (\Omega^\ast(M), d)$ be the de~Rham complex of a smooth manifold $M$.
The wedge product $\wedge$ and the exterior derivative $d$
turn $\Omega^\ast(M)$ into a commutative dga.
Then Proposition~\ref{prop:cohomology-graded-algebra} implies that
the de~Rham cohomology
\[
H^\ast_{\mathrm{dR}}(M)
   = \frac{\ker(d : \Omega^\ast(M)\to\Omega^{\ast+1}(M))}
          {\operatorname{im}(d : \Omega^{\ast-1}(M)\to\Omega^\ast(M))}
\]
is a graded-commutative algebra under the product induced by~$\wedge$.
This is the familiar ``cup product'' of smooth cohomology.
\end{example}

\begin{example}[Singular cochains]
For any topological space $X$, the singular cochain complex
$C^\ast(X;A) = \Hom_\mathbb{Z}(C_\ast(X),A)$
with differential $d$ and the cup product $\smile$
is a commutative dga over $A$.
Hence $H^\ast(X;A)$ is a graded-commutative algebra,
recovering the standard cohomology ring of topology.
\end{example}

\begin{example}[Koszul complex]
Let $A$ be a commutative ring and
$a_1,\ldots,a_n \in A$.
The Koszul complex
$K^\ast(A; a_1,\ldots,a_n) =
  \big(\bigwedge_A(e_1,\ldots,e_n), d\big)$
with $d(e_i)=a_i$ is a commutative dga.
The induced multiplication on
$H^\ast(K^\ast(A;a_1,\ldots,a_n))$
is the natural wedge product on cohomology classes.
\end{example}

\textbf{Counterexamples and Caveats}

\begin{example}[Non-dga case]
If $(C^\ast,d)$ is merely a complex of $A$-modules without a product
satisfying the Leibniz rule,
then one cannot define a ring structure on $H^\ast(C^\ast)$.
For instance, let $C^\ast = A[x]$ with $\deg x=1$ and $d(x)=1$.
Then $d^2=0$ but $d(x^2)=2x \neq 0$, violating the Leibniz rule.
Hence $(C^\ast,d)$ is \emph{not} a dga and its cohomology has no algebra structure.
\end{example}

\begin{example}[Non-commutative case]
Let $C^\ast = (M_2(\mathbb{R}) \oplus M_2(\mathbb{R})[1], d)$
with $d(A)=BA-AB$ for some fixed matrix $B$.
Then $d^2=0$ and $(C^\ast,d)$ is a (non-commutative) dga.
Its cohomology $H^\ast(C^\ast)$ is a \emph{graded algebra},
but not graded-commutative, because matrix multiplication is non-commutative.
\end{example}

\begin{example}[Failure of well-definedness without the Leibniz rule]
If the differential does not satisfy $d(uv)=d(u)v+(-1)^{\deg u}u\,d(v)$,
then the product on cohomology is not well defined:
the product of cocycles may fail to be a cocycle,
and changing representatives may change the class.
\end{example}

\textbf{Conceptual Summary}

\begin{itemize}
  \item The differential $d$ filters out ``exact'' elements from the algebra,
        leaving a graded structure of \emph{closed} classes.
  \item The Leibniz rule ensures compatibility of $d$ with multiplication,
        guaranteeing that the quotient by exact elements is again an algebra.
  \item In geometry, this explains why cohomology rings (e.g.\ de~Rham, singular,
        or \v{C}ech) have a natural multiplicative structure.
  \item In homological algebra, this property is crucial:
        $H^\ast(C^\ast)$ carries algebraic operations that survive to homotopy level,
        making it the ``first-order invariant'' of a dga.
\end{itemize}

In summary, the proposition is not merely a formal observation:
it ensures that the algebraic and differential structures
interact coherently enough for their ``homological shadow''
$H^\ast(C^\ast)$ to remain an algebra.
This compatibility is the starting point for the rich theory
of derived and homotopical algebraic structures.

\subsection{Tensor products and monoidal structures}\label{subsec:monoidal}


It will be useful to formalize the presence of a ``tensor product'' and a ``unit object''
together with the usual algebraic laws (associativity, unitality, commutativity up to coherent isomorphisms).

\begin{definition}[Symmetric monoidal category]\label{def:symm-monoidal}
A \emph{monoidal category} is the data
\[
\bigl(\mathcal{C}, \otimes, \mathbf{1}, \alpha,\lambda,\rho\bigr)
\]
where $\mathcal{C}$ is a category, $\otimes:\mathcal{C}\times\mathcal{C}\to\mathcal{C}$ is a bifunctor,
$\mathbf{1}\in\mathcal{C}$ is the \emph{unit object}, and
\begin{itemize}
  \item $\alpha_{X,Y,Z}:(X\otimes Y)\otimes Z \xrightarrow{\ \cong\ } X\otimes(Y\otimes Z)$ (associator),
  \item $\lambda_X:\mathbf{1}\otimes X \xrightarrow{\ \cong\ } X$ and $\rho_X:X\otimes \mathbf{1}\xrightarrow{\ \cong\ } X$ (left/right unitors),
\end{itemize}
are natural isomorphisms satisfying Mac~Lane's \emph{coherence axioms}:
the \emph{pentagon} for $\alpha$ and the \emph{triangle} for $(\alpha,\lambda,\rho)$.
A \emph{symmetric} monoidal category is a monoidal category endowed with a natural isomorphism
\[
\sigma_{X,Y}:X\otimes Y \xrightarrow{\ \cong\ } Y\otimes X
\]
(the \emph{symmetry} or braiding) satisfying $\sigma_{Y,X}\circ \sigma_{X,Y}=\mathrm{id}_{X\otimes Y}$ and the standard hexagon coherence.
\end{definition}

\begin{example}[Sets]\label{ex:sets-monoidal}
The category $\mathbf{Set}$ with cartesian product $X\times Y$ and unit the singleton $\{\ast\}$ is symmetric monoidal.
The constraints $\alpha,\lambda,\rho,\sigma$ are the evident bijections; Mac~Lane coherence reduces to tautological set equalities.
\end{example}

\begin{example}[Modules, vector spaces, algebras]\label{ex:modules-monoidal}
Let $R$ be a commutative ring. The categories $\mathbf{Ab}$, $R\text{-}\mathbf{Mod}$, and $R\text{-}\mathbf{Alg}$ become symmetric monoidal
under the usual tensor product $\otimes_R$, with unit object $R$ (viewed in degree~$0$ for algebras).
The associator/unitors are the canonical universal isomorphisms; symmetry is the flip $x\otimes y \mapsto y\otimes x$.
\end{example}

\begin{example}[Levelwise tensor on simplicial objects]\label{ex:sC-monoidal}
If $(\mathcal{C},\otimes,\mathbf{1})$ is (symmetric) monoidal, then the category of simplicial objects $s\mathcal{C}=\mathrm{Fun}(\Delta^{\mathrm{op}},\mathcal{C})$
is (symmetric) monoidal via the \emph{levelwise} tensor:
\[
(X\otimes Y)_n := X_n \otimes Y_n, \qquad (\mathbf{1})_n := \mathbf{1}.
\]
The faces/degeneracies act componentwise and the constraints $(\alpha,\lambda,\rho,\sigma)$ are inherited levelwise from $\mathcal{C}$.
\end{example}

\begin{proof}[Proof (Example~\ref{ex:sC-monoidal})]
Define $(X\otimes Y)(\theta) := X(\theta)\otimes Y(\theta)$ for $\theta:[m]\to[n]$ in $\Delta$. Functoriality holds since
$(X(\theta_1\circ\theta_2)\otimes Y(\theta_1\circ\theta_2))=(X(\theta_1)\circ X(\theta_2))\otimes (Y(\theta_1)\circ Y(\theta_2))$
and the bifunctor axiom yields $(f_1\circ f_2)\otimes(g_1\circ g_2)=(f_1\otimes g_1)\circ(f_2\otimes g_2)$.
Set $(\alpha_{X,Y,Z})_n := \alpha_{X_n,Y_n,Z_n}$, and similarly for $\lambda,\rho,\sigma$.
Naturality and the coherence diagrams in $s\mathcal{C}$ commute levelwise because they commute in $\mathcal{C}$.
Thus $s\mathcal{C}$ inherits a symmetric monoidal structure.
\end{proof}

\begin{example}[Chain complexes and the Koszul tensor]\label{ex:complexes-monoidal}
Let $R$ be a commutative ring and $\Ch(R)$ the category of (cochain) complexes of $R$–modules.
For $X^\ast,Y^\ast\in\Ch(R)$ define the \emph{total} tensor complex $(X^\ast\otimes_R Y^\ast,d)$ by
\[
(X\otimes Y)^n := \bigoplus_{i+j=n} X^i \otimes_R Y^j,
\qquad
d(x\otimes y) := d_X x \otimes y \;+\; (-1)^{|x|}\, x \otimes d_Y y,
\]
with unit object $R$ concentrated in degree $0$.
\end{example}

\begin{proposition}\label{prop:koszul-d2}
With the differential above, $d^2=0$, so $X^\ast\otimes_R Y^\ast$ is a complex.
Moreover, $\Ch(R)$ becomes a symmetric monoidal category with:
\begin{align*}
\alpha_{X,Y,Z}&:\ (X\otimes Y)\otimes Z \xrightarrow{\ \cong\ } X\otimes (Y\otimes Z),\\
\lambda_X&:\ R\otimes X \xrightarrow{\ \cong\ } X,\qquad
\rho_X:\ X\otimes R \xrightarrow{\ \cong\ } X,\\
\sigma_{X,Y}&:\ X\otimes Y \xrightarrow{\ \cong\ } Y\otimes X,\quad
\sigma(x\otimes y) := (-1)^{|x||y|}\, y\otimes x,
\end{align*}
each a chain isomorphism (the \emph{Koszul sign rule}).
\end{proposition}

\begin{proof}
\textbf{(i) $d^2=0$.}
Let $x\in X^{|x|}$ and $y\in Y^{|y|}$. Then
\[
\begin{aligned}
d^2(x\otimes y)
&= d\bigl(d_X x \otimes y + (-1)^{|x|} x\otimes d_Y y\bigr)\\
&= d_X^2 x \otimes y \;+\; (-1)^{|x|+1} d_X x \otimes d_Y y \;+\; (-1)^{|x|} d_X x \otimes d_Y y \;+\; (-1)^{2|x|} x \otimes d_Y^2 y\\
&= 0+ \bigl((-1)^{|x|+1}+(-1)^{|x|}\bigr)\, d_X x\otimes d_Y y + 0 \;=\; 0,
\end{aligned}
\]
since $d_X^2=d_Y^2=0$ and the middle signs cancel.

\smallskip
\textbf{(ii) Monoidal constraints.}
At the graded level, $\alpha,\lambda,\rho$ are the canonical universal isomorphisms of (graded) $R$–modules.
We must check they are \emph{chain maps}, i.e.\ commute with the differentials.
For $\lambda_X: R\otimes X \to X$, note $R$ is concentrated in degree $0$ and $d_R=0$, so
$d\circ \lambda(r\otimes x)=d_X(rx) = r\,d_X x = \lambda\bigl(d_R r\otimes x + (-1)^{0} r\otimes d_X x\bigr)=\lambda\circ d(r\otimes x)$.
Similar for $\rho$.
For $\alpha$, write $((x\otimes y)\otimes z)\mapsto x\otimes (y\otimes z)$; both sides’ differentials expand with the Koszul rule and match termwise.

\smallskip
\textbf{(iii) Symmetry and Koszul sign.}
Define $\sigma(x\otimes y)=(-1)^{|x||y|} y\otimes x$.
Then
\[
\begin{aligned}
d\,\sigma(x\otimes y)
&= d\bigl((-1)^{|x||y|} y\otimes x\bigr)
= (-1)^{|x||y|}\bigl(d_Y y\otimes x + (-1)^{|y|} y\otimes d_X x\bigr),\\
\sigma\,d(x\otimes y)
&= \sigma\bigl(d_X x\otimes y + (-1)^{|x|} x\otimes d_Y y\bigr)\\
&= (-1)^{(|x|+1)|y|} y\otimes d_X x \;+\; (-1)^{|x|(|y|+1)} d_Y y\otimes x\\
&= (-1)^{|x||y|+|y|} y\otimes d_X x \;+\; (-1)^{|x||y|} d_Y y\otimes x,
\end{aligned}
\]
which equals $d\,\sigma(x\otimes y)$. Hence $\sigma$ is a chain isomorphism and $(\Ch(R),\otimes,R,\alpha,\lambda,\rho,\sigma)$ is symmetric monoidal.
\end{proof}

\begin{remark}[Analyst's view: why the signs?]\label{rmk:koszul-why}
The Koszul rule encodes graded-commutativity \emph{at the chain level} so that:
(i) the total differential squares to zero, and
(ii) the symmetry $\sigma$ is a \emph{chain} map (not just graded).
These are precisely the conditions needed for the homotopy category $\mathrm{D}(R)$
to inherit a symmetric monoidal structure where $X^\ast\otimes^\mathbf{L} Y^\ast$ is computed by projective (or flat) resolutions.
\end{remark}

\begin{remark}[Compatibility with cohomology]
If $X^\ast,Y^\ast$ are dg–algebras (resp.\ cdga’s), then $X^\ast\otimes_R Y^\ast$
is naturally a dg–algebra (resp.\ cdga) with product
$(x\otimes y)\cdot(x'\otimes y') := (-1)^{|y||x'|}(xx')\otimes(yy')$.
The differential satisfies Leibniz by construction, and the symmetry is again Koszul.
Passing to cohomology yields the usual cup product on $H^\ast(X^\ast)\otimes H^\ast(Y^\ast)$.
\end{remark}

\subsection{Monoid objects in a monoidal category}\label{subsec:monoid-objects}

\begin{definition}[Monoid object]\label{def:monoid-object}
Let $(\mathcal{C},\otimes,\mathbf{1},\alpha,\lambda,\rho)$ be a monoidal category.
A \emph{monoid object} (or simply a \emph{monoid}) is a triple $(M,\mu,\eta)$ where
$M\in\mathcal{C}$, $\mu:M\otimes M\to M$ (multiplication) and $\eta:\mathbf{1}\to M$ (unit)
are morphisms in $\mathcal{C}$ such that the following diagrams commute:


\[\begin{tikzcd}
(M\otimes M)\otimes M \ar[r, "\alpha_{M,M,M}"] \ar[d, "\mu\otimes \Id_M"'] &
M\otimes (M\otimes M) \ar[d, "\Id_M\otimes \mu"] \\
M\otimes M \ar[r, "\mu"'] & M
\end{tikzcd}\]

\[\begin{tikzcd}
\mathbf{1}\otimes M \ar[r, "\lambda_M"] \ar[dr, "\eta\otimes \Id_M"'] & M \\
& M\otimes M \ar[u, "\mu"']
\end{tikzcd}\]

\[\begin{tikzcd}
M\otimes \mathbf{1} \ar[r, "\rho_M"] \ar[dr, "\Id_M\otimes \eta"'] & M \\
& M\otimes M \ar[u, "\mu"']
\end{tikzcd}\]

The first is \emph{associativity}; the other two are \emph{left/right unitality}.
\end{definition}

\begin{definition}[Commutative (symmetric) monoid object]\label{def:commutative-monoid}
If $(\mathcal{C},\otimes,\mathbf{1},\sigma)$ is symmetric monoidal, a monoid $(M,\mu,\eta)$ is
\emph{commutative} if
\[
\mu \;=\; \mu\circ \sigma_{M,M}: M\otimes M \longrightarrow M,
\]
i.e.\ $\mu$ is compatible with the symmetry (the \emph{twist}).
\end{definition}

\begin{remark}[Algebra objects]
A (commutative) monoid object in $\mathcal{C}$ is often called a (commutative) \emph{algebra object} of $\mathcal{C}$.
The data $\mu,\eta$ are morphisms in $\mathcal{C}$, so they must respect whatever structure morphisms carry in $\mathcal{C}$ (e.g.\ grading, differentials, etc.).
\end{remark}

\begin{example}[Recovering ordinary monoids]\label{ex:monoid-in-Set}
In $(\mathbf{Set},\times,\{\ast\})$, a monoid object is precisely an ordinary monoid:
$\mu: M\times M\to M$ is a binary operation, $\eta:\{\ast\}\to M$ picks out the unit $e$,
the associativity diagram says $(ab)c=a(bc)$, and the unit diagrams say $ea=a=ae$.
\end{example}

\begin{example}[Algebras over a ring]\label{ex:algebra-object-in-RMod}
Let $R$ be a commutative ring. In $(R\text{-}\mathbf{Mod},\otimes_R,R)$, a monoid object $(A,\mu,\eta)$
is an associative unital $R$–algebra:
$\mu: A\otimes_R A\to A$ is $R$–bilinear multiplication and $\eta:R\to A$ picks the unit $1_A=\eta(1)$.
If the monoidal structure is symmetric, the commutativity condition $\mu=\mu\circ \sigma$ is exactly
$ab=ba$ for all $a,b\in A$.
\end{example}

\begin{example}[DG–algebras as monoid objects]\label{ex:dga-as-monoid-object}
In $(\Ch(R),\otimes,R)$ (Example~\ref{ex:complexes-monoidal}), a monoid object $(C^\ast,\mu,\eta)$
is precisely a \emph{differential graded algebra} (dga):
\begin{itemize}
  \item $\mu$ is a morphism of complexes, hence a chain map of degree $0$, so for homogeneous $x,y$
  \[
  d\bigl(\mu(x\otimes y)\bigr)\;=\;\mu\bigl(d(x\otimes y)\bigr)
  \;=\;\mu(d x\otimes y + (-1)^{|x|} x\otimes d y),
  \]
  i.e.\ the Leibniz rule $d(xy)=d(x)\,y+(-1)^{|x|}x\,d(y)$ holds;
  \item the associativity and unital diagrams give associativity and a unit in the graded sense.
\end{itemize}
If moreover $(\Ch(R),\otimes,R,\sigma)$ is taken with the Koszul symmetry
$\sigma(x\otimes y)=(-1)^{|x||y|}y\otimes x$, then a \emph{commutative} monoid object is a \emph{cdga}:
$xy=(-1)^{|x||y|}yx$.
\end{example}

\begin{example}[Simplicial monoids]\label{ex:simplicial-monoid}
If $(\mathcal{C},\otimes,\mathbf{1})$ is monoidal, then by Example~\ref{ex:sC-monoidal}, $(s\mathcal{C},\otimes,\mathbf{1})$ is monoidal levelwise. A monoid object in $s\mathcal{C}$ is a simplicial object $M_\bullet$ together with face/degeneracy-compatible
$\mu_\bullet: M_\bullet\otimes M_\bullet\to M_\bullet$ and $\eta_\bullet:\mathbf{1}\to M_\bullet$, i.e.\ a \emph{simplicial monoid}.
In particular, simplicial rings are monoid objects in $s(R\text{-}\mathbf{Mod})$.
\end{example}

\begin{proposition}[Morphisms and forgetful functor]\label{prop:monoid-morphisms}
A morphism of monoid objects $f:(M,\mu,\eta)\to (N,\nu,\eta')$ in $\mathcal{C}$ is a morphism $f:M\to N$ such that
\[
f\circ \mu \;=\; \nu\circ (f\otimes f)
\qquad\text{and}\qquad
f\circ \eta \;=\; \eta'.
\]
Monoid objects and their morphisms form a category $\Mon(\mathcal{C})$, and the forgetful functor
$U:\Mon(\mathcal{C})\to \mathcal{C}$ is faithful and creates limits that exist in $\mathcal{C}$.
\end{proposition}

\begin{proof}[Verification]
Faithfulness of $U$ is immediate. Limits in $\Mon(\mathcal{C})$ are computed in $\mathcal{C}$ with the induced pointwise multiplication and unit, checked by the universal property (standard). The equations defining a morphism encode multiplicativity and unit-preservation, matching the usual algebra homomorphism axioms in Examples~\ref{ex:algebra-object-in-RMod} and~\ref{ex:dga-as-monoid-object}.
\end{proof}

\begin{remark}[Coherence and uniqueness of parenthesization]
By Mac~Lane's coherence theorem, in a monoidal category every diagram built from $\alpha,\lambda,\rho$ that should commute \emph{does} commute. Consequently, for a monoid object the two composites $M^{\otimes n}\to M$ obtained by multiplying $n$ factors are uniquely determined (independent of parenthesization).
\end{remark}

\begin{remark}[Commutative vs.\ symmetric]
In a merely braided monoidal category, one can still define \emph{commutative} monoids via the braiding; in a symmetric monoidal category this reduces to the usual symmetry and coincides with the standard notion of a commutative algebra object.
\end{remark}

\begin{remark}[Why this abstraction matters]
This formalism lets us speak uniformly about rings, algebras, graded algebras, dg–algebras, sheaves of algebras, $E_\infty$–algebras, etc., as \emph{monoids} inside different ambient monoidal categories. It is the language that makes derived and homotopical constructions functorial and invariant under equivalences.
\end{remark}

\textbf{Examples of monoid objects}\label{subsec:monoid-examples}

\begin{example}[Monoids in \texorpdfstring{$\mathbf{Set}$}{Set}]\label{ex:monoid-in-Set}
A monoid object in the symmetric monoidal category $(\mathbf{Set},\times,\{\ast\})$
is precisely an ordinary algebraic monoid.

Indeed, let $(M,\mu,\eta)$ be such a monoid object.  
Then $\eta:\{\ast\}\to M$ selects an element $e:=\eta(\ast)\in M$,  
and $\mu:M\times M\to M$ defines a binary operation
\[
(x,y)\longmapsto \mu(x,y)=:xy.
\]
The commutative diagrams of Definition~\ref{def:monoid-object} express exactly
the usual axioms:
\[
(xy)z = x(yz),
\qquad
ex = x = xe,
\qquad \forall\,x,y,z\in M.
\]
Conversely, every monoid $(M,\cdot,e)$ in the classical sense
determines a monoid object in $(\mathbf{Set},\times,\{\ast\})$
by setting $\mu(x,y)=x\cdot y$ and $\eta(\ast)=e$.
\end{example}

\begin{proof}[Verification]
Associativity in the diagram
\[
(M\times M)\times M \xrightarrow{\ \mu\times \Id\ } M\times M \xrightarrow{\ \mu\ } M
\quad=\quad
M\times (M\times M) \xrightarrow{\ \Id\times \mu\ } M\times M \xrightarrow{\ \mu\ } M
\]
means $(xy)z=x(yz)$.
The left and right unit diagrams translate to
$\mu(e,x)=x=\mu(x,e)$.
Hence the categorical and algebraic notions coincide.
\end{proof}

\begin{remark}
This example shows that the categorical definition of monoid object
is a direct abstraction of the classical algebraic one.
In this sense, monoid objects \emph{generalize} monoids to any monoidal setting.
\end{remark}

% ---------------------------------------------------------------

\begin{example}[Monoids in abelian groups and modules]\label{ex:monoid-in-Ab}
In the monoidal category $(\mathbf{Ab},\otimes_\mathbb{Z},\mathbb{Z})$
a monoid object $(R,\mu,\eta)$ consists of:
\begin{itemize}
  \item a multiplication $\mu:R\otimes_\mathbb{Z}R\to R$, i.e.\ a bilinear map $(x,y)\mapsto xy$;
  \item a unit $\eta:\mathbb{Z}\to R$ corresponding to an element $1_R=\eta(1)$.
\end{itemize}
The diagrams for associativity and unitality state that $(R,\mu,\eta)$
is an associative unital ring.  
If $R$–\textbf{Mod} replaces $\mathbf{Ab}$, a monoid object is an \emph{$R$–algebra}.
\end{example}

\begin{proof}[Verification]
Bilinearity follows from the tensor product’s universal property.
Associativity and the unit axioms hold because $\mu$ and $\eta$
satisfy the same diagrams as in the classical definition of a ring
and $R$–algebra.  Conversely, any associative unital ring defines a monoid object
by the canonical maps
\(\mu(x\otimes y)=xy,\ \eta(1)=1_R.\)
\end{proof}

\begin{remark}
The passage from sets to abelian groups (or modules) replaces
the cartesian product by the tensor product, ensuring bilinearity.
This shows that the monoidal structure encodes the kind of
``linearity'' required of the multiplication.
\end{remark}

% ---------------------------------------------------------------

\begin{example}[Simplicial monoids]\label{ex:monoid-in-sC}
Let $(\mathcal{C},\otimes,\mathbf{1})$ be a monoidal category.
Then the category $s\mathcal{C}$ of simplicial objects inherits
a levelwise monoidal structure (Example~\ref{ex:sC-monoidal}).
A monoid object $(M_\bullet,\mu_\bullet,\eta_\bullet)$ in $s\mathcal{C}$ is
a simplicial object in $\mathcal{C}$ such that each level
$M_n$ carries a monoid structure in $\mathcal{C}$ and
the face and degeneracy maps
\[
d_i : M_n \longrightarrow M_{n-1},
\qquad
s_j : M_n \longrightarrow M_{n+1}
\]
are morphisms of monoids.
Thus $M_\bullet$ is the \emph{simplicial version} of a monoid in $\mathcal{C}$.
For example, simplicial rings are monoid objects in $s(R\text{-}\mathbf{Mod})$.
\end{example}

\begin{proof}[Verification]
Because $(M\otimes N)_n = M_n\otimes N_n$,
the multiplication $\mu_\bullet$ and unit $\eta_\bullet$ must be collections
of morphisms $\mu_n:M_n\otimes M_n\to M_n$, $\eta_n:\mathbf{1}\to M_n$
that satisfy the same coherence diagrams at each level.
Naturality of the face and degeneracy maps ensures that all these
multiplications are compatible across simplicial degrees.
\end{proof}

\begin{remark}
This categorical perspective provides a unifying framework for 
\emph{simplicial algebras}, \emph{simplicial rings}, or \emph{simplicial groups}:
they are simply monoid (or group) objects in $s\mathcal{C}$.
\end{remark}

% ---------------------------------------------------------------

\begin{example}[Differential graded algebras]\label{ex:monoid-in-Ch}
Let $R$ be a commutative ring and consider the symmetric monoidal category
$(\Ch(R),\otimes,R,\sigma)$ of chain complexes of $R$–modules
with the Koszul sign rule $\sigma(x\otimes y)=(-1)^{|x||y|}y\otimes x$.
A monoid object $(C^\ast,\mu,\eta)$ in this category is
a \emph{differential graded algebra (dga)}:
\begin{itemize}
  \item $\mu:C^\ast\otimes C^\ast\to C^\ast$ is a morphism of complexes,
        which is equivalent to the Leibniz rule
        \[
        d(xy)=d(x)\,y + (-1)^{|x|}x\,d(y);
        \]
  \item $\eta:R\to C^\ast$ picks the unit element $1\in C^0$;
  \item associativity and unit diagrams ensure $xy\cdot z=x\cdot(yz)$ and $1x=x=x1$.
\end{itemize}
If $\mu=\mu\circ\sigma$, then $xy=(-1)^{|x||y|}yx$, making $C^\ast$ a
\emph{commutative differential graded algebra (cdga)}.
\end{example}

\begin{proof}[Verification]
Because $\mu$ is a morphism of complexes, $d\circ \mu = \mu\circ d_{C\otimes C}$.
Expanding $d_{C\otimes C}(x\otimes y)=d(x)\otimes y+(-1)^{|x|}x\otimes d(y)$
gives $d(xy)=d(x)y+(-1)^{|x|}x\,d(y)$.
The associativity and unit diagrams are inherited directly from the monoidal structure,
and the symmetry condition $\mu=\mu\circ\sigma$ yields graded commutativity.
\end{proof}

\begin{remark}
This example shows that all of the ``algebraic'' objects encountered so far—
monoids, rings, algebras, dg–algebras—are instances of a single universal notion:
a \emph{monoid object} in a suitable monoidal category.
\end{remark}

\begin{example}[Derived Tensor Product]\label{ex:derived-tensor}
Let $R$ be a commutative ring and let $\mathbf{D}(R)$ denote the (unbounded) derived
category of complexes of $R$–modules, obtained from the homotopy category by inverting quasi-isomorphisms.
We endow $\mathbf{D}(R)$ with a symmetric monoidal structure via the \emph{derived tensor product}
\[
-\otimes_R^{\mathbf L}- \;:\; \mathbf{D}(R)\times \mathbf{D}(R)\longrightarrow \mathbf{D}(R).
\]

\paragraph{Models (resolutions).}
There are several equivalent constructions. For the unbounded derived category, the most robust one uses
\emph{K-flat} or \emph{K-projective} resolutions.

\begin{definition}[K-flat and K-projective complexes]
A complex $F^\ast$ of $R$–modules is \emph{K-flat} if for every acyclic complex $A^\ast$, the tensor complex
$F^\ast\otimes_R A^\ast$ is acyclic. A complex $P^\ast$ is \emph{K-projective} if
$\Hom_{\mathrm{Ch}(R)}(P^\ast,-)$ sends quasi-isomorphisms to quasi-isomorphisms (equivalently,
$\Hom(P^\ast, A^\ast)$ is acyclic for every acyclic $A^\ast$).
\end{definition}

\begin{remark}
Every complex admits a quasi-isomorphism from a K-projective complex and a quasi-isomorphism to a K-flat complex
(Spaltenstein's theorem). Bounded below complexes of flat (resp.\ projective) modules are K-flat (resp.\ K-projective).
\end{remark}

\paragraph{Definition of $A\otimes_R^{\mathbf L}B$.}
Given $A,B\in \mathbf{D}(R)$, choose a K-flat resolution $Q^\ast \xrightarrow{\ \simeq\ } B$ (or K-projective resolution
$P^\ast \xrightarrow{\ \simeq\ } A$). Define
\[
A\otimes_R^{\mathbf L} B \ :=\ A \otimes_R Q^\ast \ \in \mathbf{D}(R)
\qquad
\text{(or equivalently $P^\ast\otimes_R B$ or $P^\ast\otimes_R Q^\ast$).}
\]
This yields a bifunctor well-defined up to canonical isomorphism in $\mathbf{D}(R)$.

\paragraph{Well-definedness.}
\end{example}

\begin{proposition}[Independence of resolutions]\label{prop:derived-tensor-welldef}
Let $Q_1^\ast \xrightarrow{\simeq} B$ and $Q_2^\ast \xrightarrow{\simeq} B$ be K-flat resolutions. Then
$A\otimes_R Q_1^\ast \xrightarrow{\ \sim\ } A\otimes_R Q_2^\ast$ is a quasi-isomorphism, naturally in $A$.
Similarly, if $P_1^\ast\xrightarrow{\simeq} A$ and $P_2^\ast\xrightarrow{\simeq} A$ are K-projective, then
$P_1^\ast\otimes_R B \xrightarrow{\ \sim\ } P_2^\ast\otimes_R B$ is a quasi-isomorphism, naturally in $B$.
\end{proposition}

\begin{proof}[Proof sketch]
Consider a zigzag $Q_1^\ast \xleftarrow{\simeq} Z^\ast \xrightarrow{\simeq} Q_2^\ast$ in the homotopy category
connecting the two K-flat resolutions (obtained by a factorization/roof). Tensoring with any $A$ preserves
quasi-isomorphisms because $Q_i^\ast$ are K-flat and tensoring with $A$ preserves quasi-isomorphisms between
K-flat complexes. Hence $A\otimes_R Q_1^\ast \leftarrow A\otimes_R Z^\ast \to A\otimes_R Q_2^\ast$ is a zigzag
of quasi-isomorphisms. The K-projective case is dual: applying $\Hom(P_i^\ast,-)$ preserves quasi-isomorphisms,
and by adjunction one reduces to the K-flat argument.
\end{proof}

\begin{proposition}[Bifunctoriality and symmetry]\label{prop:derived-tensor-bifunctor}
The operation $-\otimes_R^{\mathbf L}-$ descends to a bifunctor on $\mathbf{D}(R)$, with unit object $R$
(concentrated in degree $0$), associative and symmetric up to canonical isomorphism. Thus $(\mathbf{D}(R),\otimes_R^{\mathbf L},R)$
is a symmetric monoidal category.
\end{proposition}

\begin{proof}[Proof sketch]
Choose K-flat (or K-projective) resolutions functorially (up to homotopy). The associativity and symmetry constraints
are induced from the usual tensor product on complexes together with the Koszul sign rule; independence of choices follows
from Proposition~\ref{prop:derived-tensor-welldef}. The unit constraints are induced by $R\otimes_R^{\mathbf L} A\simeq A \simeq A\otimes_R^{\mathbf L} R$,
since $R$ is K-flat and the underived tensor with $R$ agrees with the identity.
\end{proof}

\begin{theorem}[Homology and $\Tor$]\label{thm:Tor-from-derived-tensor}
For $A,B\in \mathbf{D}(R)$ (or for ordinary $R$–modules viewed as complexes concentrated in degree~$0$), there are natural isomorphisms
\[
H_i\bigl(A\otimes_R^{\mathbf L} B\bigr) \;\cong\; \Tor^R_i(A,B)\qquad (i\in \mathbb{Z}),
\]
in particular $H_0\bigl(A\otimes_R^{\mathbf L} B\bigr)\cong A\otimes_R B$.
\end{theorem}

\begin{proof}[Proof sketch]
Take a K-flat resolution $Q^\ast\xrightarrow{\simeq} B$. Then $A\otimes_R^{\mathbf L} B \simeq A\otimes_R Q^\ast$,
and the homology of $A\otimes_R Q^\ast$ computes the left-derived functors of $A\otimes_R-$ evaluated at $B$.
When $A,B$ are ordinary modules in degree $0$, this recovers the classical definition of $\Tor$ via a flat (or projective) resolution of one input.
The $H_0$ statement follows because $Q^\ast \to B$ is a quasi-isomorphism and the degree-$0$ homology of $A\otimes_R Q^\ast$ is $A\otimes_R B$.
\end{proof}

\begin{remark}[Indexing conventions]
We use homological grading, so $H_i$ sits in (homological) degree $i\le 0$ when starting with nonnegative cochain degrees.
If cohomological grading is preferred, one writes $H^{-i}(A^\bullet\otimes_R^{\mathbf L} B^\bullet) \cong \Tor^R_i(A^\bullet,B^\bullet)$.
\end{remark}

\begin{remark}[Resolving one factor suffices]
If either $A$ or $B$ is K-flat, then $A\otimes_R^{\mathbf L} B \simeq A\otimes_R B$ (no resolution of the other factor needed).
For bounded-below complexes of flat modules, the ordinary tensor already computes the derived tensor.
\end{remark}

\begin{remark}[Compatibility with algebra structures]
If $A^\ast$ and $B^\ast$ are (commutative) dg–algebras over $R$, then $A^\ast\otimes_R^{\mathbf L} B^\ast$ acquires a natural structure
of (commutative) dg–algebra in $\mathbf{D}(R)$; on cohomology this yields the familiar K\"unneth multiplicative structure
under the Koszul sign rule.
\end{remark}

\subsection{From simplicial rings to differential graded algebras}\label{subsec:simplicial-to-dga}

Let $\mathcal{A}$ be an abelian category endowed with a symmetric monoidal structure
$(\otimes, \mathbf{1}, \alpha, \lambda, \rho, \sigma)$.
Both the category of simplicial objects $s\mathcal{A} = \Fun(\Delta^{\mathrm{op}}, \mathcal{A})$
and the category of non-negatively graded chain complexes $\Ch_{\ge 0}(\mathcal{A})$
inherit symmetric monoidal structures from that of $\mathcal{A}$:
\[
(X\otimes Y)_n = X_n\otimes Y_n, \qquad
(X\otimes Y)_n = \bigoplus_{p+q=n} X_p\otimes Y_q
\]
respectively for the simplicial and chain versions.

By the \emph{Dold–Kan correspondence}, these two categories are equivalent as abelian categories:
\[
N: s\mathcal{A} \xrightarrow{\ \sim\ } \Ch_{\ge 0}(\mathcal{A}),
\qquad
\Gamma: \Ch_{\ge 0}(\mathcal{A}) \xrightarrow{\ \sim\ } s\mathcal{A},
\]
where $N$ is the \emph{normalization functor} and $\Gamma$ its inverse.

However, the tensor products on $s\mathcal{A}$ and on $\Ch_{\ge 0}(\mathcal{A})$
are \emph{not} transported to each other under this equivalence:
the normalization functor $N$ does not preserve the monoidal structure.
This subtle but important fact leads to the appearance of the \emph{shuffle product} on normalized complexes.

\begin{remark}[Two monoidal structures]
If we transport the tensor product of simplicial objects along the Dold–Kan equivalence,
we obtain a \emph{different} monoidal structure on $\Ch_{\ge 0}(\mathcal{A})$,
called the \emph{shuffle product} (or \emph{Eilenberg–Zilber tensor product}).
It is naturally isomorphic to the ordinary tensor product only up to homotopy,
and is generally larger (more terms appear in each degree).
\end{remark}

This phenomenon is controlled by the classical \emph{Eilenberg–Zilber theorem},
which compares the diagonal and total complexes of bisimplicial objects.

% ---------------------------------------------------
\begin{definition}[Bisimplicial object]\label{def:bisimplicial-object}
A \emph{bisimplicial object} in a category $\mathcal{C}$ is a functor
\[
A: (\Delta\times \Delta)^{\mathrm{op}} \longrightarrow \mathcal{C},
\]
that is, a simplicial object in the category of simplicial objects of $\mathcal{C}$:
\[
A \in s(s\mathcal{C}) = \Fun(\Delta^{\mathrm{op}}, \Fun(\Delta^{\mathrm{op}}, \mathcal{C})).
\]
Equivalently, it is a doubly indexed collection $\{A_{p,q}\}_{p,q\ge 0}$ of objects in $\mathcal{C}$
equipped with \emph{horizontal} and \emph{vertical} face and degeneracy maps
\[
h\partial_i: A_{p,q}\to A_{p-1,q}, \quad h\sigma_i:A_{p,q}\to A_{p+1,q},
\qquad
v\partial_i:A_{p,q}\to A_{p,q-1}, \quad v\sigma_i:A_{p,q}\to A_{p,q+1},
\]
satisfying the simplicial identities in each direction,
and such that all horizontal maps commute with all vertical ones.
\end{definition}

\begin{example}[Double complex from a bisimplicial object]\label{ex:double-complex}
If $\mathcal{A}$ is an abelian category, any bisimplicial object $A_{\bullet,\bullet}$ gives rise to
a first-quadrant double complex
\[
A_{p,q}\in \mathcal{A}, \qquad p,q\ge 0,
\]
with horizontal differential
\[
d_h = \sum_{i=0}^{p} (-1)^i\, h\partial_i: A_{p,q} \longrightarrow A_{p-1,q}
\]
and vertical differential
\[
d_v = \sum_{i=0}^{q} (-1)^{p+i}\, v\partial_i: A_{p,q} \longrightarrow A_{p,q-1}.
\]
These signs ensure that $d_h^2=d_v^2=0$ and $d_hd_v+d_vd_h=0$,
so that the total differential $d = d_h + d_v$ on the total complex
$\mathrm{Tot}(A)_{n} = \bigoplus_{p+q=n} A_{p,q}$ satisfies $d^2=0$.
\end{example}

\begin{definition}[Diagonal functor]\label{def:diag-functor}
The \emph{diagonal functor}
\[
\diag: ss\mathcal{A} \longrightarrow s\mathcal{A}
\]
is defined by precomposition with the diagonal embedding
$\Delta \to \Delta\times\Delta$.
Explicitly,
\[
(\diag A)_n := A_{n,n}, \qquad
\partial_i = h\partial_i \circ v\partial_i, \quad
\sigma_i = h\sigma_i \circ v\sigma_i.
\]
Thus $\diag(A)$ is the simplicial object obtained by restricting $A$ along the diagonal direction.
\end{definition}

\begin{remark}[Comparison: diagonal vs.\ total complex]
For a bisimplicial abelian group $A_{\bullet,\bullet}$, there are two canonical simplicial (or chain) objects associated to it:
\begin{enumerate}
  \item The \emph{diagonal} simplicial object $\diag(A)$;
  \item The \emph{total complex} $\Tot(A)$ obtained from the double complex in Example~\ref{ex:double-complex}.
\end{enumerate}
They are not isomorphic but are related by a natural chain homotopy equivalence
provided by the Eilenberg–Zilber theorem.
\end{remark}

% ---------------------------------------------------
\begin{theorem}[Eilenberg–Zilber]\label{thm:eilenberg-zilber}
Let $\mathcal{A}$ be an abelian category and $A_{\bullet,\bullet}$ a bisimplicial object in $\mathcal{A}$.
Then the normalization of the diagonal and the total complex are canonically chain-homotopy equivalent:
\[
N(\diag A) \ \simeq\  \Tot(N_h N_v A),
\]
where $N_h$ and $N_v$ denote the horizontal and vertical normalizations, respectively.
Explicitly, there exist natural chain maps
\[
\mathrm{AW}: N(\diag A) \longrightarrow \Tot(N_hN_v A),
\qquad
\mathrm{Sh}: \Tot(N_hN_v A) \longrightarrow N(\diag A)
\]
(the \emph{Alexander–Whitney} and \emph{shuffle} maps)
such that both composites $\mathrm{AW}\circ\mathrm{Sh}$ and $\mathrm{Sh}\circ\mathrm{AW}$
are homotopic to the identity.
\end{theorem}

\begin{proof}[Sketch of proof]
The Alexander–Whitney map splits a simplex into its ``front'' and ``back'' faces,
whereas the shuffle map combines faces of two simplices in all possible interleavings
respecting order. Both are constructed explicitly at the level of chains
and are natural transformations between the two constructions.
Eilenberg and Mac~Lane proved that these transformations induce chain homotopy equivalences,
see~\cite[Chap.~VIII]{Weibel} or~\cite[Thm.~1.6.4]{GoerssJardine}.
\end{proof}

\begin{remark}[Consequences for simplicial rings and cdga’s]
By applying the normalization functor $N$ to a simplicial commutative ring $A_\bullet$,
we obtain a nonnegatively graded commutative differential graded algebra:
\[
N(A_\bullet) \in \mathrm{cdga}_{\ge 0}(R).
\]
The product on $N(A_\bullet)$ is induced via the shuffle map of the Eilenberg–Zilber theorem.
This correspondence underlies the passage from simplicial rings to cdga’s,
which becomes an equivalence of homotopy categories after suitable localization
(e.g.\ Quillen’s equivalence between simplicial commutative rings and connective cdga’s).
\end{remark}

\begin{theorem}[Eilenberg--Zilber]\label{thm:EZ}
Let $\mathcal{A}$ be an abelian category and let $A_{\bullet,\bullet}$ be a bisimplicial object in $\mathcal{A}$.
Then there is a natural isomorphism
\[
\pi_\ast\bigl(\diag A_{\bullet,\bullet}\bigr)\ \cong\ H_\ast\bigl(\Tot(N_hN_v A_{\bullet,\bullet})\bigr),
\]
where $\diag(A)_n:=A_{n,n}$ is the diagonal simplicial object, $N_h,N_v$ are horizontal/vertical normalizations,
and $\Tot$ is the total complex of the resulting first–quadrant double complex.
Equivalently, there is a natural chain homotopy equivalence
\[
N\bigl(\diag A\bigr)\ \simeq\ \Tot\bigl(N_hN_v A\bigr),
\]
so that $\pi_\ast(\diag A)=H_\ast\bigl(N(\diag A)\bigr)\cong H_\ast\bigl(\Tot(N_hN_v A)\bigr)$.
\end{theorem}

\begin{proof}[Proof (explicit maps and homotopies)]
We recall the classical Alexander--Whitney and shuffle maps.

\paragraph{Alexander--Whitney map.}
Define $\mathrm{AW}_n: A_{n,n}\to \bigoplus_{p+q=n} A_{p,q}$ by
\[
\mathrm{AW}_n(a)\ :=\ \sum_{p+q=n}\ \bigl(h\partial_{p+1}\cdots h\partial_{n}\bigr)\ \bigl(v\partial_{0}\bigr)^{q}(a),
\qquad a\in A_{n,n}.
\]
Here $(v\partial_0)^q$ means $q$ successive applications of $v\partial_0$, and $h\partial_{p+1}\cdots h\partial_n$
are the horizontal faces applied from right to left (so the empty product is the identity when $p=n$ or $q=0$).
Passing to normalizations gives a well-defined degree–preserving map
\[
\mathrm{AW}: N(\diag A)\ \longrightarrow\ \Tot\bigl(N_hN_vA\bigr).
\]

\paragraph{Shuffle map.}
For each $p,q\ge 0$ with $p+q=n$, let $\mathrm{Sh}(p,q)$ be the set of $(p,q)$–shuffles,
i.e.\ pairs of strictly increasing functions $\mu:\{1,\dots,p\}\hookrightarrow \{1,\dots,n\}$,
$\nu:\{1,\dots,q\}\hookrightarrow \{1,\dots,n\}$ with disjoint images whose union is $\{1,\dots,n\}$.
Write $\sgn(\mu,\nu)$ for the parity of the permutation that interleaves the $\mu$– and $\nu$–slots.
Define $\mathrm{Sh}_{p,q}:A_{p,q}\to A_{n,n}$ by
\[
\mathrm{Sh}_{p,q}(x)\ :=\ \sum_{(\mu,\nu)\in \mathrm{Sh}(p,q)} (-1)^{\sgn(\mu,\nu)}
\ \Bigl(h\sigma_{\nu_q}\cdots h\sigma_{\nu_1}\Bigr)\ \Bigl(v\sigma_{\mu_p}\cdots v\sigma_{\mu_1}\Bigr)(x),
\]
where $h\sigma_i$ (resp.\ $v\sigma_i$) are horizontal (resp.\ vertical) degeneracies.
Summing over $p+q=n$ and passing to normalizations yields a degree–preserving map
\[
\mathrm{Sh}: \Tot\bigl(N_hN_vA\bigr)\ \longrightarrow\ N(\diag A).
\]

\paragraph{Chain map property.}
A standard (but careful) sign computation shows that both $\mathrm{AW}$ and $\mathrm{Sh}$
commute with the differentials:
\[
d\,\mathrm{AW}=\mathrm{AW}\,d,\qquad d\,\mathrm{Sh}=\mathrm{Sh}\,d.
\]
The key ingredients are: (i) simplicial identities, (ii) the definitions of $d_h$ and $d_v$
(with the $(-1)^{p+i}$ sign in $d_v$), and (iii) book-keeping of front/back faces vs.\ interleavings.

\paragraph{Homotopy inverses.}
There exist explicit natural chain homotopies $H_1, H_2$ (constructed by Eilenberg--Mac~Lane)
such that
\[
\mathrm{AW}\circ \mathrm{Sh}\ \simeq\ \Id_{\Tot(N_hN_vA)},
\qquad
\mathrm{Sh}\circ \mathrm{AW}\ \simeq\ \Id_{N(\diag A)}.
\]
Therefore $N(\diag A)$ and $\Tot(N_hN_v A)$ are naturally chain homotopy equivalent.
Taking homology gives the displayed isomorphism, and using Dold--Kan, $\pi_\ast(\diag A)=H_\ast(N(\diag A))$.
\end{proof}

\begin{corollary}[K\"unneth-type comparison for simplicial tensor]\label{cor:Kun-simplicial}
Let $R$ be a commutative ring and let $A_\bullet,B_\bullet$ be simplicial $R$–modules.
Consider the bisimplicial object $C_{m,n}:=A_m\otimes_R B_n$.
Then
\[
\pi_\ast\bigl(\diag(A_\bullet\otimes_R B_\bullet)\bigr)\ \cong\ H_\ast\Bigl(\Tot\bigl(NA_\bullet\ \widetilde{\otimes}_R\ NB_\bullet\bigr)\Bigr),
\]
where $\diag(A_\bullet\otimes_R B_\bullet)$ is the simplicial tensor $n\mapsto A_n\otimes_R B_n$,
and $\Tot\bigl(NA_\bullet\ \widetilde{\otimes}_R\ NB_\bullet\bigr)$ is canonically isomorphic to
the usual tensor product of chain complexes $NA_\bullet\otimes_R NB_\bullet$ (totalized with Koszul sign).
Equivalently,
\[
\pi_\ast\bigl(\diag(A_\bullet\otimes_R B_\bullet)\bigr)\ \cong\ H_\ast\bigl(NA_\bullet\otimes_R NB_\bullet\bigr).
\]
\end{corollary}

\begin{proof}
Apply Theorem~\ref{thm:EZ} to $C_{m,n}=A_m\otimes_R B_n$. The double complex from $N_hN_v$ is precisely
the bicomplex whose totalization is the tensor product of the normalized complexes $NA_\bullet$ and $NB_\bullet$.
\end{proof}

\begin{remark}[Classical shuffle formula on simplicial tensors]
In the situation of the corollary, the shuffle map on the diagonal $A_n\otimes_R B_n$
has the concrete form
\[
\mathrm{Sh}_n(a\otimes b)\ =\ \sum_{p+q=n}\ \sum_{(\mu,\nu)\in\mathrm{Sh}(p,q)}
(-1)^{\sgn(\mu,\nu)}\ \Bigl(\sigma_{\nu_q}\cdots \sigma_{\nu_1}a\Bigr)\ \otimes\
\Bigl(\sigma_{\mu_p}\cdots \sigma_{\mu_1}b\Bigr),
\]
i.e.\ a signed sum over all $(p,q)$–interleavings of degeneracies; this is the standard
Eilenberg--Mac~Lane shuffle. Dually, the Alexander--Whitney map splits faces into
front/back parts. These induce mutually inverse isomorphisms on homotopy/homology.
\end{remark}

\begin{remark}[Why this matters for Dold--Kan]
Dold--Kan yields $N:s\mathcal{A}\overset{\sim}{\to}\Ch_{\ge 0}(\mathcal{A})$ but \emph{not} monoidally:
the simplicial tensor and the chain tensor do not match on the nose. The Eilenberg--Zilber
equivalences above show they agree \emph{up to canonical chain homotopy}, hence induce the same
(co)homology. This is exactly what is used to pass from simplicial rings to cdga’s via the
shuffle multiplication on normalized chains.
\end{remark}

\subsection{Shuffle and Alexander--Whitney maps}\label{subsec:AW-shuffle}

Throughout let $\mathcal{A}$ be an abelian category. For a simplicial object
$A_\bullet\in s\mathcal{A}$ write faces $d_i:A_n\to A_{n-1}$ and degeneracies $s_i:A_n\to A_{n+1}$,
$0\le i\le n$.

\begin{definition}[(\emph{$(p,q)$}-shuffles and sign)]\label{def:shuffle-sign}
For $p,q\ge 0$, a \emph{$(p,q)$-shuffle} is a pair of strictly increasing maps
\[
\mu:\{1,\dots,p\}\hookrightarrow \{1,\dots,p+q\},
\qquad
\nu:\{1,\dots,q\}\hookrightarrow \{1,\dots,p+q\},
\]
with disjoint images whose union is $\{1,\dots,p+q\}$.
It determines a permutation $\sigma(\mu,\nu)\in \Sigma_{p+q}$ that places the
$\mu$-slots first and the $\nu$-slots second, and we denote by
\(
(-1)^{\sgn(\mu,\nu)}
\)
the sign of that permutation.
\end{definition}

\begin{definition}[Shuffle map]\label{def:shuffle-map}
Let $A_\bullet,B_\bullet\in s\mathcal{A}$. The (Eilenberg--Mac~Lane) \emph{shuffle map}
\[
\nabla:\ N(A_\bullet)\ \otimes\ N(B_\bullet)\ \longrightarrow\ N(A_\bullet\otimes B_\bullet)
\]
is defined on homogeneous generators $a\in N_pA_\bullet\subseteq A_p$ and
$b\in N_qB_\bullet\subseteq B_q$ by
\[
\nabla(a\otimes b)
\;=\;
\sum_{(\mu,\nu)\in \mathrm{Sh}(p,q)}
(-1)^{\sgn(\mu,\nu)}\;
\Bigl(s_{\nu_q}\cdots s_{\nu_1}a\Bigr)\ \otimes\
\Bigl(s_{\mu_p}\cdots s_{\mu_1}b\Bigr)
\ \in\ (A\otimes B)_{p+q},
\]
and then projected to the normalized subgroup $N_{p+q}(A\otimes B)$.
\end{definition}

\begin{definition}[Alexander--Whitney map]\label{def:AW-map}
The \emph{Alexander--Whitney map}
\[
\mathrm{AW}:\ N(A_\bullet\otimes B_\bullet)\ \longrightarrow\ N(A_\bullet)\ \otimes\ N(B_\bullet)
\]
is the degree-preserving map whose $n$-th component
$\mathrm{AW}_n:(A\otimes B)_n\to \bigoplus_{p+q=n}A_p\otimes B_q$ is
\[
\mathrm{AW}_n(a\otimes b)
\;=\;
\sum_{p+q=n}\;
\Bigl(d_{p+1}\cdots d_n\,a\Bigr)\ \otimes\ \Bigl(d_0\cdots d_{p-1}\,b\Bigr),
\qquad a\in A_n,\ b\in B_n,
\]
followed by the projection to $N_pA\otimes N_qB$ in each summand.
\end{definition}

\begin{remark}[Front/back faces and degeneracies]
In the AW formula, $d_{p+1}\cdots d_n$ takes the \emph{front} $p$-face of $a$,
while $d_0\cdots d_{p-1}$ takes the \emph{back} $q$-face of $b$ (with $q=n-p$).
In the shuffle formula, $s_{\nu_\bullet}$ and $s_{\mu_\bullet}$ distribute degeneracies
to interleave degrees $p$ and $q$ into degree $p+q$.
\end{remark}

\begin{proposition}[Both maps factor through normalization]\label{prop:factor-normalized}
The formulas in Definitions~\ref{def:shuffle-map} and~\ref{def:AW-map} send degenerate chains
to degenerate chains; hence they induce well-defined maps on normalized complexes as stated.
\end{proposition}

\begin{proof}
For $\nabla$, if $a=s_i(a')$ is degenerate in $A_p$, then each summand contains
$s_{\nu_q}\cdots s_{\nu_1}s_i(a')$ in the $A$-slot, hence is degenerate in $(A\otimes B)_{p+q}$;
similarly if $b$ is degenerate. For $\mathrm{AW}$, if $a$ or $b$ is degenerate, at least one
of $d_{p+1}\cdots d_n a$ or $d_0\cdots d_{p-1} b$ is degenerate (by simplicial identities),
so the image lies in the degenerate subcomplex. Details are standard and use
$d_is_j=\begin{cases}s_{j-1}d_i & i<j,\\ \mathrm{id} & i=j\text{ or }i=j+1,\\ s_jd_{i-1} & i>j+1.\end{cases}$
\end{proof}

\begin{proposition}[Chain map property]\label{prop:chain-maps}
With the standard simplicial differential $d=\sum_{i=0}^n(-1)^i d_i$, both
\[
d\,\nabla=\nabla\, d
\qquad\text{on }N(A)\otimes N(B),
\qquad\text{and}\qquad
d\,\mathrm{AW}=\mathrm{AW}\, d
\quad\text{on }N(A\otimes B).
\]
\end{proposition}

\begin{proof}[Sketch]
For $\nabla$, compute $d$ of each shuffle summand using $d_is_j$ identities and track the sign
$(-1)^{\sgn(\mu,\nu)}$; terms cancel in pairs corresponding to interleavings where one face hits a
degeneracy in two different orders. For $\mathrm{AW}$, expand $d$ on $a\otimes b$ and compare with
$d$ on each front/back component; the simplicial relations again split contributions into front vs.\ back
and the signs match the AW summation indices. Standard references carry the full bookkeeping.
\end{proof}

\begin{theorem}[Eilenberg--Zilber equivalence]\label{thm:EZ-equivalence}
The maps
\[
\nabla:\ N(A)\otimes N(B)\longrightarrow N(A\otimes B),
\qquad
\mathrm{AW}:\ N(A\otimes B)\longrightarrow N(A)\otimes N(B),
\]
are natural chain maps such that
\[
\mathrm{AW}\circ \nabla\ =\ \mathrm{id}_{N(A)\otimes N(B)},
\qquad
\nabla\circ \mathrm{AW}\ \simeq\ \mathrm{id}_{N(A\otimes B)}
\]
(chain homotopy equivalence). In particular, both are quasi-isomorphisms.
\end{theorem}

\begin{proof}
On normalized chains, $\mathrm{AW}\circ\nabla=\Id$:
apply $\mathrm{AW}$ to a single shuffle summand; only the \emph{identity} shuffle contributes a
nondegenerate term (others become degenerate and vanish in $N(-)$).
Hence the composite is exactly $a\otimes b$.
For the other composite, Eilenberg--Mac~Lane construct an explicit prism homotopy
$H: N_n(A\otimes B)\to N_{n+1}(A\otimes B)$ satisfying
$dH+Hd=\Id-\nabla\circ\mathrm{AW}$. The formula inserts one extra degeneracy and sums
over positions with appropriate signs; verification uses the same identities as above.
Therefore $\nabla$ and $\mathrm{AW}$ are homotopy inverses.
\end{proof}

\begin{remark}[Symmetric monoidal behavior]\label{rmk:monoidal}
The shuffle map endows the normalization functor
$N:s\mathcal{A}\to \Ch_{\ge 0}(\mathcal{A})$ with the structure of a \emph{symmetric lax monoidal}
functor:
\[
\nabla_{A,B}:\ N(A)\otimes N(B)\longrightarrow N(A\otimes B),
\qquad
N(\mathbf{1})\cong \mathbf{1},
\]
compatible with associativity, unit, and symmetry isomorphisms
(the symmetry is the Koszul sign, matching $(-1)^{pq}$ on $N_pA\otimes N_qB$).
By contrast, $\mathrm{AW}$ is \emph{colax} monoidal but generally not symmetric (it is not compatible with the twist).
\end{remark}

\begin{remark}[Low-degree sanity check]
For $p=q=0$, $\nabla(a\otimes b)=a\otimes b$ and $\mathrm{AW}(a\otimes b)=a\otimes b$.
For $p=1,q=0$,
$\nabla(a_1\otimes b_0)=s_1 a_1\otimes b_0 + s_0 a_1\otimes b_0$, and
$\mathrm{AW}(a_1\otimes b_1)=d_1 a_1\otimes b_1 + a_1\otimes d_0 b_1$,
illustrating ``front/back'' splitting vs.\ interleaving degeneracies.
\end{remark}

\subsection{From simplicial rings to differential graded algebras}\label{subsec:srings-to-dgas}

We continue with an abelian symmetric monoidal category $(\mathcal{A},\otimes,\mathbf{1})$.
In particular, take $\mathcal{A}=R\text{-}\mathbf{Mod}$ for a commutative ring $R$.

Recall (Definitions~\ref{def:shuffle-map} and~\ref{def:AW-map}) that the
\emph{normalization} functor
\[
N:\ s\mathcal{A}\longrightarrow \Ch_{\ge 0}(\mathcal{A})
\]
is equipped with a \emph{symmetric lax monoidal} structure
\[
\nabla_{A,B}:\ N(A)\otimes N(B)\longrightarrow N(A\otimes B),\qquad N(\mathbf{1})\cong \mathbf{1},
\]
induced by the \emph{shuffle map}; and the \emph{denormalization} (quasi-inverse) functor
\[
K:\ \Ch_{\ge 0}(\mathcal{A}) \longrightarrow s\mathcal{A}
\]
is equipped with a \emph{colax monoidal} structure
\[
\mathrm{AW}_{A,B}:\ K(A\otimes B)\longrightarrow K(A)\otimes K(B),
\qquad K(\mathbf{1})\cong \mathbf{1},
\]
induced by the \emph{Alexander--Whitney map}. The former is symmetric, the latter is generally \emph{not}.

\begin{proposition}\label{prop:12020}
\emph{(Cf.\ \S I.3.1.3)} The functors $N$ and $K$ induce the following functors between algebra objects:
\[
N:\ s\mathbf{CRing}\ \longrightarrow\ s(\mathbf{cdga}),
\qquad
K:\ s(\mathbf{dga})\ \longrightarrow\ s\mathbf{Ring}.
\]
Here $s\mathbf{CRing}$ denotes simplicial commutative rings, $\mathbf{dga}$ (resp.\ $\mathbf{cdga}$)
denotes (commutative) differential graded algebras in $\Ch_{\ge 0}(R\text{-}\mathbf{Mod})$, and
$s(\mathbf{dga})$ means simplicial objects valued in $\mathbf{dga}$.
\end{proposition}

\begin{proof}[Proof (functorial mechanism)]
A lax monoidal functor sends \emph{monoid objects} to monoid objects.
Since $N$ is \emph{symmetric} lax monoidal via $\nabla$,
it carries \emph{commutative} monoids in $s(R\text{-}\mathbf{Mod})$ (i.e.\ simplicial commutative rings)
to \emph{commutative} monoids in $\Ch_{\ge 0}(R\text{-}\mathbf{Mod})$, i.e.\ cdga's,
and this levelwise for simplicial objects, giving $N:\ s\mathbf{CRing}\to s(\mathbf{cdga})$.

Conversely, $K$ is \emph{colax} monoidal via $\mathrm{AW}$.
Passing to \emph{noncommutative} monoids, a colax structure on $K$ yields (by adjunction with $N$)
a natural multiplication on $K$–images of dga's which is associative and unital,
but $K$ is not \emph{symmetric}, so commutativity does not transport.
Thus we obtain $K:\ s(\mathbf{dga})\to s\mathbf{Ring}$ but not a functor
$s(\mathbf{cdga})\to s\mathbf{CRing}$ in general.
\end{proof}

\begin{remark}[Why cdga’s appear from simplicial commutative rings]\label{rmk:cdga-from-scr}
The multiplication on $N(A_\bullet)$ for a simplicial commutative ring $A_\bullet$
is defined by transporting the simplicial tensor along the \emph{shuffle}:
on normalized chains, it is \emph{graded commutative} with the Koszul sign
$xy=(-1)^{|x||y|}yx$. Hence $N(A_\bullet)$ is a cdga concentrated in nonnegative degrees.
\end{remark}

\begin{remark}[No symmetric structure for $K$]\label{rmk:K-not-symmetric}
The Alexander--Whitney map $\mathrm{AW}_{A,B}$ fails to commute with the symmetry (twist),
so $K$ does not acquire a \emph{symmetric} lax monoidal structure.
Consequently, there is no canonical functor $K:\ s(\mathbf{cdga})\to s\mathbf{CRing}$.
\end{remark}

\begin{proposition}[Odd squares and divided powers]\label{prop:odd-square}
Let $A_\bullet$ be a simplicial commutative ring. Then the cohomology (homology) algebra
$H^\ast(N A_\bullet)$ (or $H_\ast(N A_\bullet)$, depending on grading convention)
is a graded-commutative algebra endowed with natural \emph{divided power} operations.
In particular, for any homogeneous class $x$ of odd degree,
\[
x^2=0.
\]
\end{proposition}

\begin{proof}[Explanation]
Simplicial commutative rings carry \emph{Dyer--Lashof/divided power} operations at the chain level,
and the shuffle product on $N(A_\bullet)$ interacts with degeneracies to produce a \emph{divided power structure}
$\{\gamma_k\}$ on homology. In a divided power algebra, one has $\gamma_2(x)=\frac{x^2}{2}$ and $\gamma_2(x)=0$ for $|x|$ odd,
which forces $x^2=0$ integrally. For a general graded-commutative dga (with no divided powers) one only has
$2x^2=0$ when $|x|$ is odd, coming from the Koszul sign $x^2=(-1)^{|x|^2}x^2=-x^2$.
\end{proof}

\begin{corollary}[Rectification obstruction in positive/unspecified characteristic]\label{cor:rectification}
Without a characteristic zero hypothesis, not every cdga is quasi-isomorphic to $N(A_\bullet)$ for some simplicial commutative ring $A_\bullet$:
the ``odd squares vanish'' constraint above holds for $H^\ast(NA_\bullet)$ but not for a general cdga (where only $2x^2=0$ can be guaranteed).
More generally, $H^\ast(NA_\bullet)$ has divided powers that need not exist on the homology of an arbitrary cdga.
\end{corollary}

\begin{remark}[Model structures and the need for $E_\infty$]\label{rmk:model-structures}
The forgetful functor $\mathbf{dga}\to \Ch_{\ge 0}(R\text{-}\mathbf{Mod})$ does not (in general) \emph{create} a model structure:
there is no naive way to transfer the model structure from complexes to dga’s that is homotopically well-behaved and preserves commutativity.
In arbitrary characteristic, the homotopically meaningful replacement for commutative algebra objects is the category of \emph{$E_\infty$–algebras}
(commutative up to all coherent higher homotopies). In characteristic zero, rectification often allows an equivalence
$E_\infty\text{-}\mathbf{Alg}\simeq \mathbf{cdga}$, but this fails in general in positive characteristic.
\end{remark}
\subsection{From simplicial cdga's to a single cdga}\label{subsec:simplicial-cdga-to-cdga}

\begin{lemma}[Illusie’s shuffle totalization]\label{lem:tot-cdga}
There is a functor
\[
\Tot_{\shuffle}:\ \Fun(\Delta^{\op},\mathbf{cdga}_A)\ \longrightarrow\ \mathbf{cdga}_A,
\]
sending a simplicial differential graded $A$–algebra to a (nonnegatively) graded
commutative differential graded $A$–algebra.
\end{lemma}

\begin{proof}
Let $(B^\ast_\bullet,d,\wedge)$ be a simplicial cdga over a commutative ring $A$.
Write $B^p_i$ for the component in simplicial degree $p\ge 0$ and cochain degree $i\ge 0$.
Face/degeneracy maps in the simplicial direction are cdga morphisms and commute with $d$.

\smallskip
\textbf{(1) Bicomplex and total complex.}
Form the first-quadrant bicomplex with horizontal (simplicial) and vertical (cochain) directions
\[
\begin{tikzcd}[ampersand replacement=\&]
\ddots \& \vdots \& \vdots \\
\cdots \arrow[r] \& B^{p}_{i+1} \arrow[r,"\,\partial\,"] \arrow[u,"d"] \& B^{p-1}_{i+1} \arrow[u,"d"] \\
\cdots \arrow[r] \& B^{p}_{i} \arrow[r,"\,\partial\,"] \arrow[u,"d"] \& B^{p-1}_{i} \arrow[u,"d"] \\
\cdots \arrow[r] \& B^{p}_{i-1} \arrow[r,"\,\partial\,"] \arrow[u,"d"] \& B^{p-1}_{i-1} \arrow[u,"d"]
\end{tikzcd}
\]
where $\partial=\sum_{r=0}^{p}(-1)^r d_r$ is the alternating simplicial face differential
($d_r$ the $r$-th face), and $d:B^p_i\to B^p_{i+1}$ is the internal cochain differential.

Define the \emph{total} cochain complex $\Tot_\oplus(B)$ by
\[
\Tot_\oplus^n(B)\ :=\ \bigoplus_{\,p-i=n} B^p_i,\qquad
D\,\big|_{B^p_i}\ :=\ \partial\ +\ (-1)^{p}\,d.
\]
\emph{Claim:} $D^2=0$. Indeed, $\partial^2=0$, $d^2=0$, and $d$ commutes with each face $d_r$
(because faces are cdga morphisms), so $\partial d=d\,\partial$. Hence
\[
D^2=\partial^2 + (-1)^p(\partial d + d\partial) + (-1)^{2p} d^2=0.
\]

\smallskip
\textbf{(2) Shuffle product on $\Tot_\oplus(B)$.}
For homogeneous $\omega\in B^p_i$ and $\eta\in B^q_j$, define
\[
\omega\ \shuffle\ \eta
\;:=\;
(-1)^{p\,j}\!\!\sum_{(\mu,\nu)\in\Sh(p,q)}\!\!
(-1)^{\sgn(\mu,\nu)}\;
\bigl(s_{\nu_q}\cdots s_{\nu_1}\,\omega\bigr)\ \wedge\ \bigl(s_{\mu_p}\cdots s_{\mu_1}\,\eta\bigr)
\ \in\ B^{p+q}_{\,i+j},
\tag{$\ast$}\label{eq:shuffle-product}
\]
where $\Sh(p,q)$ is the set of $(p,q)$–shuffles, $\sgn(\mu,\nu)$ the parity of the interleaving permutation,
and $s_k$ are degeneracies in the simplicial direction. The factor $(-1)^{p\,j}$ is the
\emph{Illusie sign} (cf.\ Remark~\ref{rmk:Illusie-sign}).

Extend bilinearly to $\Tot_\oplus^\ast(B)\otimes \Tot_\oplus^\ast(B)\to \Tot_\oplus^\ast(B)$.
Let $1\in B^0_0$ be the multiplicative unit (constant simplices); it serves as unit for $\shuffle$.

\smallskip
\textbf{(3) Associativity.}
Associativity of $\shuffle$ follows from: (i) associativity of the shuffle map in the simplicial
direction (Eilenberg--Zilber), (ii) associativity of $\wedge$ in each level, and
(iii) functoriality of degeneracies. Precisely,
\[
(\omega\shuffle\eta)\shuffle\zeta
\ =\
\omega\shuffle(\eta\shuffle\zeta),
\]
with the same total sign obtained by counting the Koszul/permution signs in both parenthesizations.

\smallskip
\textbf{(4) Graded commutativity.}
Let $|\omega|=p-i$ and $|\eta|=q-j$ be the \emph{total} degrees. Then
\[
\eta\shuffle\omega
\;=\;
(-1)^{|\omega|\,|\eta|}\,\omega\shuffle\eta.
\]
Indeed, swapping the roles of $(p,i)$ and $(q,j)$ flips the shuffle by
$(-1)^{pq}$ (parity of $(p,q)$–interleavings), the internal cdga commutativity contributes
$(-1)^{ij}$, and the Illusie sign contributes $(-1)^{q i}$ vs.\ $(-1)^{p j}$.
Altogether one gets $(-1)^{pq+ij+qi+pj}=(-1)^{(p-i)(q-j)}$.

\smallskip
\textbf{(5) Leibniz rule for $D$.}
We show
\[
D(\omega\shuffle\eta)
\;=\;
D\omega\shuffle\eta\ +\ (-1)^{|\omega|}\,\omega\shuffle D\eta.
\]
Split $D=\partial+(-1)^p d$ on $B^p_i$ and $D=\partial+(-1)^q d$ on $B^q_j$.
The $\partial$–part uses that the shuffle map is a chain map with respect to the simplicial differential
(Eilenberg--Zilber: $d\,\nabla=\nabla\,d$), together with the bookkeeping of the Illusie sign
$(-1)^{p j}$ in~\eqref{eq:shuffle-product}.
The $d$–part uses that each face/degeneracy is a cdga morphism (so commutes with $d$), and that
$d$ is a derivation for $\wedge$:
\[
d\bigl((s_\nu\omega)\wedge (s_\mu\eta)\bigr)
=(ds_\nu\omega)\wedge(s_\mu\eta)+(-1)^i (s_\nu\omega)\wedge(ds_\mu\eta).
\]
Combining the signs (including $(-1)^p$ and $(-1)^q$ from $D$) yields exactly $(-1)^{|\omega|}$ in the second term.

\smallskip
\textbf{(6) Unit.}
The element $1\in B^0_0\subset \Tot_\oplus^0(B)$ satisfies $1\shuffle x=x=x\shuffle 1$:
only the identity shuffle contributes, degeneracies of $1$ are again $1$, and the Illusie sign is $1$.

\smallskip
\textbf{(7) Functoriality.}
A morphism $f_\bullet^\ast:B_\bullet^\ast\to C_\bullet^\ast$ in $\Fun(\Delta^{\op},\mathbf{cdga}_A)$
is levelwise a cdga map commuting with faces/degeneracies. Apply $f$ to each summand
in~\eqref{eq:shuffle-product} to see $f$ preserves $\shuffle$ and $D$, hence induces a cdga map
$\Tot_{\shuffle}(B)\to \Tot_{\shuffle}(C)$. Thus $\Tot_{\shuffle}$ is functorial.

\smallskip
Therefore $\bigl(\Tot_\oplus^\ast(B),D,\shuffle,1\bigr)$ is a commutative dga over $A$,
natural in $B$. Set $\Tot_{\shuffle}(B):=\bigl(\Tot_\oplus^\ast(B),D,\shuffle,1\bigr)$.
\end{proof}

\begin{remark}[On the Illusie sign]\label{rmk:Illusie-sign}
The leading factor $(-1)^{p\,j}$ in~\eqref{eq:shuffle-product} is essential to make the Leibniz rule
and graded commutativity hold with the total degree $|\,\cdot\,|=p-i$.
(We thank T.~Szamuely and G.~Z\'abr\'adi for highlighting this choice.)
It aligns the simplicial (homological) and internal (cohomological) gradings in the totalization.
\end{remark}

\begin{proof}[Detailed proof of the Leibniz rule and functoriality]
Fix a simplicial cdga $(B^\ast_\bullet,d,\wedge)$ over $A$.
Recall the total complex
\[
\Tot^n(B)\;=\;\bigoplus_{p-i=n} B^p_i,
\qquad
D\big|_{B^p_i}=\partial+(-1)^p d,
\]
and the product (Illusie shuffle)
\begin{equation}\label{eq:Illusie-prod}
\omega\cdot\eta
\;=\;(-1)^{pj}\!\!\sum_{(\mu,\nu)\in \Sh(p,q)}\!\!
(-1)^{\sgn(\mu,\nu)}\ \sigma_\nu(\omega)\ \wedge_{i+j}\ \sigma_\mu(\eta),
\qquad
\omega\in B^p_i,\ \eta\in B^q_j,
\end{equation}
where $\sigma_\nu=s_{\nu_q}\cdots s_{\nu_1}$ and $\sigma_\mu=s_{\mu_p}\cdots s_{\mu_1}$ are the
simplicial degeneracies and $\wedge_{i+j}$ is the multiplication in cochain degree $i+j$.
We must show:
\[
D(\omega\cdot\eta)\;=\;D\omega\cdot\eta\;+\;(-1)^{p-i}\ \omega\cdot D\eta.
\]

\medskip

\noindent\textbf{1) Reduction to a $2\times 2$ square.}
Write $|\omega|=p-i$ and $|\eta|=q-j$. Since $D=\partial+(-1)^p d$ on $B^p_i$ and
$D=\partial+(-1)^q d$ on $B^q_j$, it suffices to prove the commutativity of the square
\[
\begin{tikzcd}[row sep=large, column sep=large]
B^p_i\otimes_A B^q_j \ar[r,"{\ \cdot\ }"] \ar[d,"{\partial_i+(-1)^p d^p_i\ +\ (-1)^{p-i}\bigl(-\partial_j+(-1)^q d^q_j\bigr)}"']
&
B^{p+q}_{i+j} \ar[d,"{\ \partial_{i+j}\ +\ (-1)^{i+j} d_{i+j}^{p+q}\ }"]
\\
B^{p-1}_i\otimes B^q_j \ \oplus\ B^p_{i+1}\otimes B^q_j \ \oplus\ B^p_i\otimes B^{q-1}_j \ \oplus\ B^p_i\otimes B^q_{j+1}
\ar[r,"{\ \cdot\ }"']
&
B^{p+q-1}_{i+j} \ \oplus\ B^{p+q}_{i+j+1}
\end{tikzcd}
\]
(the decomposition simply records where each summand of $D$ lands). On homogeneous
$\omega\in B^p_i$, $\eta\in B^q_j$, this commutativity is equivalent to the pair of identities
\begin{align}
(-1)^{pj}\!\!\sum_{(\mu,\nu)}\!\! \sgn(\mu,\nu)\,
\partial_{i+j}\bigl(\sigma_\nu\omega \ \wedge_{i+j}\ \sigma_\mu\eta\bigr)
&=
(-1)^{pj}\!\!\sum_{(\mu',\nu')}\!\!\sgn(\mu',\nu')\,
\sigma_{\nu'}(\partial_i\omega) \ \wedge_{i+j-1}\ \sigma_{\mu'}\eta
\label{eq:front-face}\\[-1pt]
&\quad
+(-1)^{pj}(-1)^{p-i}\!\!\sum_{(\mu'',\nu'')}\!\!\sgn(\mu'',\nu'')\,
\sigma_{\nu''}\omega \ \wedge_{i+j-1}\ \sigma_{\mu''}(\partial_j\eta),
\tag{1.4--1.5}
\\[4pt]
(-1)^{pj}(-1)^{i+j}\!\!\sum_{(\mu,\nu)}\!\!\sgn(\mu,\nu)\,
d^{p+q}_{i+j}\bigl(\sigma_\nu\omega \ \wedge_{i+j}\ \sigma_\mu\eta\bigr)
&=
(-1)^{p+1}(-1)^{pj}\!\!\sum_{(\mu,\nu)}\!\!\sgn(\mu,\nu)\,
\sigma_\nu(d^p_i\omega)\ \wedge_{i+j}\ \sigma_\mu\eta
\notag\\[-2pt]
&\qquad
+(-1)^{pj}(-1)^{p-i}\!\!\sum_{(\mu,\nu)}\!\!\sgn(\mu,\nu)\,
\sigma_\nu\omega\ \wedge_{i+j}\ \sigma_\mu(d^q_j\eta).
\label{eq:internal-d}
\tag{1.6}
\end{align}
Here we wrote only those summands that survive in the indicated bidegrees; all other face terms cancel
pairwise by shuffle combinatorics.

\medskip

\noindent\textbf{2) The three elementary commutative squares.}
The equalities \eqref{eq:front-face}–\eqref{eq:internal-d} follow from the commutativity of the
following three natural squares in $A$–modules (all morphisms preserve multiplication and $d$):

\begin{equation}\label{sq:face-left}
\begin{tikzcd}[column sep=large]
B^{p-1}_i\otimes_A B^q_j \ar[r,"{\ \shuffle\ }"] \ar[d,"{\ \partial_i\ \otimes\ \Id\ }"']
&
B^{p+q-1}_{i+j-1}\otimes_A B^0_0 \ar[r,"{\ \wedge_{i+j-1}\ }"]
&
B^{p+q-1}_{\,i+j-1} \ar[d,"{\ \partial_{i+j-1}\ }"]
\\
B^p_i\otimes_A B^q_j \ar[rr,"{\ \shuffle\ }"']
&&
B^{p+q}_{\,i+j-1}
\end{tikzcd}
\end{equation}

\begin{equation}\label{sq:face-right}
\begin{tikzcd}[column sep=large]
B^{p}_i\otimes_A B^{q-1}_j \ar[r,"{\ \shuffle\ }"] \ar[d,"{\ \Id\ \otimes\ \partial_j\ }"']
&
B^{p+q-1}_{i+j-1}\otimes_A B^0_0 \ar[r,"{\ \wedge_{i+j-1}\ }"]
&
B^{p+q-1}_{\,i+j-1} \ar[d,"{\ \partial_{i+j-1}\ }"]
\\
B^p_i\otimes_A B^q_j \ar[rr,"{\ \shuffle\ }"']
&&
B^{p+q}_{\,i+j-1}
\end{tikzcd}
\end{equation}

\begin{equation}\label{sq:internal-d}
\begin{tikzcd}[column sep=large]
B^{p}_i\otimes_A B^q_j \ar[r,"{\ \shuffle\ }"] \ar[d,"{\ d^p_i\ \otimes\ \Id\ \oplus\ (-1)^i\,\Id\ \otimes\ d^q_j\ }"']
&
B^{p+q}_{i+j} \ar[d,"{\ d^{p+q}_{i+j}\ }"]
\\
B^{p}_{i+1}\otimes_A B^q_j \ \oplus\ B^{p}_i\otimes_A B^q_{j+1}
\ar[r,"{\ \shuffle\ }"']
&
B^{p+q}_{i+j+1}
\end{tikzcd}
\end{equation}

\noindent
Explanations:
\begin{itemize}
\item The \emph{shuffle} is a chain map for the simplicial differential (Eilenberg–Zilber),
so it respects faces; this gives the left/right face squares \eqref{sq:face-left}–\eqref{sq:face-right}.
\item Each face/degeneracy is a cdga morphism; in particular it commutes with $d$
and preserves $\wedge$. This and the derivation property of $d$ yield \eqref{sq:internal-d},
including the Koszul sign $(-1)^i$ on the \(\Id\otimes d\) branch.
\item The global sign $(-1)^{pj}$ in \eqref{eq:Illusie-prod} exactly corrects the parity
when we interchange the simplicial degree \(p\) and the internal degree \(j\),
so that the total degree \(|\omega|=p-i\) governs the Leibniz sign \( (-1)^{|\omega|}\).
\end{itemize}
Chasing \eqref{sq:face-left}–\eqref{sq:internal-d} gives \eqref{eq:front-face}–\eqref{eq:internal-d},
hence $D(\omega\cdot\eta)=D\omega\cdot\eta+(-1)^{|\omega|}\omega\cdot D\eta$.

\medskip

\noindent\textbf{3) Consequences.}
We have shown that $(\Tot^\ast(B),D,\cdot)$ is a dga. Associativity and graded
commutativity (with respect to total degree) follow from the associativity of $\wedge$
and the standard combinatorics of shuffles; the unit is $1\in B^0_0$.

\medskip

\noindent\textbf{4) Functoriality.}
Let $f: B^\ast_\bullet\to C^\ast_\bullet$ be a morphism in $\Fun(\Delta^{\op},\mathbf{cdga}_A)$.
For each $p$ we get a commutative cube of $A$–modules
\[
\begin{tikzcd}[row sep=small, column sep=small]
B^p_{i+1} \ar[rr,"{f^p_{i+1}}"] \ar[dd,"{d^p_{i}}"'] \ar[dr,"{\partial_i}"] && C^p_{i+1} \ar[dd,"{d^p_{i}}"] \ar[dr,"{\partial_i}"] & \\
& B^{p+1}_{i} \ar[rr, crossing over, "{f^{p+1}_{i}}"] \ar[dd,"{d^{p+1}_{i-1}}"'] && C^{p+1}_{i} \ar[dd,"{d^{p+1}_{i-1}}"] \\
B^p_{i} \ar[rr,"{f^p_{i}}"'] \ar[dr,"{\partial_{i-1}}"'] && C^p_{i} \ar[dr,"{\partial_{i-1}}"] & \\
& B^{p+1}_{i-1} \ar[rr,"{f^{p+1}_{i-1}}"'] && C^{p+1}_{i-1}
\end{tikzcd}
\]
(the faces say that $f$ commutes with faces and differentials and is multiplicative levelwise).
Define
\[
\int f\ :=\ \bigoplus_{p-i=n} f^p_i\ :\ \Tot^n(B)\longrightarrow \Tot^n(C).
\]
Then $\int f$ is a cochain map (since each square commutes) and
\[
(\int f)(\omega\cdot\eta)
= (-1)^{pj}\!\!\sum_{(\mu,\nu)}\!\!\sgn(\mu,\nu)\
f^{p+q}_{i+j}\!\bigl(\sigma_\nu\omega\ \wedge_{i+j}\ \sigma_\mu\eta\bigr)
= (-1)^{pj}\!\!\sum_{(\mu,\nu)}\!\!\sgn(\mu,\nu)\
\sigma_\nu f^p_i(\omega)\ \wedge_{i+j}\ \sigma_\mu f^q_j(\eta)
\]
(using that $f$ is a functor $\Delta^{\op}\!\to\mathbf{cdga}_A$, so it commutes with degeneracies
and preserves products). Hence
\[
(\int f)(\omega\cdot\eta) \;=\; (\int f)(\omega)\ \cdot\ (\int f)(\eta),
\]
i.e. $\int f:\Tot(B)\to\Tot(C)$ is a graded algebra map; together with the cochain check,
$\int f$ is a \emph{cdga} morphism. This proves functoriality.
\end{proof}

\subsection{Truncations of differential graded algebras}\label{subsec:truncations-cdga}

Let $(C^\ast,d,\cdot)$ be a commutative differential graded $A$–algebra (cdga),
cohomological grading (so $d:C^n\to C^{n+1}$).

\begin{definition}[Na\"\i ve (canonical) truncations of complexes]\label{def:canonical-trunc}
For any cochain complex $C^\ast$ and integer $n\in\mathbb{Z}$, set
\[
\bigl(t_{[n}C^\ast\bigr)^k :=
\begin{cases}
0 & k<n,\\
C^n/d(C^{n-1}) & k=n,\\
C^k & k>n,
\end{cases}
\qquad
\bigl(t_{n]}C^\ast\bigr)^k :=
\begin{cases}
C^k & k<n,\\
\ker(d:C^n\to C^{n+1}) & k=n,\\
0 & k>n,
\end{cases}
\]
with the evident differentials. There are canonical morphisms
\[
\pi:\ C^\ast \longrightarrow t_{[n}C^\ast,
\qquad
\iota:\ t_{n]}C^\ast \longrightarrow C^\ast.
\]
\end{definition}

\begin{remark}[Naturality on cohomology]\label{rmk:trunc-cohom-iso}
The maps in Definition~\ref{def:canonical-trunc} induce isomorphisms
\[
H^k(C^\ast)\xrightarrow{\ \cong\ } H^k\bigl(t_{[n}C^\ast\bigr)\quad (k>n),
\qquad
H^k\bigl(t_{n]}C^\ast\bigr)\xrightarrow{\ \cong\ } H^k(C^\ast)\quad (k<n).
\]
Hence, if $H^k(C^\ast)=0$ for $k<n$ (resp.\ $k>n$), then
$C^\ast\to t_{[n}C^\ast$ (resp.\ $t_{n]}C^\ast\to C^\ast$) is a quasi-isomorphism.
\end{remark}

The preceding constructions are \emph{complex-theoretic}. We now ask: do they
inherit a \emph{cdga} structure, and do the canonical maps preserve multiplication?

\paragraph{Failure of multiplicativity for the na\"\i ve truncations.}

\begin{proposition}[Na\"\i ve truncations are not cdga's in general]\label{prop:naive-fails}
In general, $t_{[n}C^\ast$ and $t_{n]}C^\ast$ do not admit a cdga structure
for which the canonical maps are cdga morphisms, unless the multiplication is
trivial for degree reasons (e.g.\ $C^\ast$ concentrated in degree $0$).
\end{proposition}

\begin{proof}
\emph{Case \(\boldsymbol{t_{[n}}\).}
In degree $n$ one works in the quotient $C^n/B^n$ with $B^n:=d(C^{n-1})$.
Take a class $[a]\in C^n/B^n$ and $b\in C^m$ ($m>0$).
If we replace $a$ by $a+d(\alpha)$, then
\[
(a+d\alpha)\cdot b \;=\; a\cdot b \;+\; d(\alpha)\cdot b
\;=\; a\cdot b \;+\; d(\alpha b)\ -\ (-1)^{|\alpha|}\,\alpha\cdot d(b).
\]
In degree $n+m$ of $t_{[n}C^\ast$ there is \emph{no quotient}, so for well-definedness we
would need $d(\alpha)\cdot b$ to be a boundary \emph{for all} $b$; but the extra term
$(-1)^{|\alpha|}\alpha\cdot d(b)$ shows this fails whenever $d(b)\neq 0$.
Thus multiplication by a degree-$n$ class is not well-defined.

\emph{Case \(\boldsymbol{t_{n]}}\).}
Here degree $n$ is \(\ker d^n\subset C^n\). Let $x\in\ker d^n$ and $y\in C^m$ with $m>0$.
Then $xy\in C^{n+m}$, but $t_{n]}C^\ast$ vanishes in degrees \(>n\); hence closure under
multiplication forces $xy=0$ in the truncated object, which is incompatible with the
canonical inclusion \(t_{n]}C^\ast\to C^\ast\) being multiplicative unless all such products
already vanish in $C^\ast$ (e.g.\ $C^{>0}=0$). Therefore one cannot make $t_{n]}$ into a
cdga with the desired property in general.
\end{proof}

The failure above explains Remark~1.2.25: degree-$n$ multiplication in $t_{[n}C^\ast$
depends on choices of representatives, unless the complex is concentrated in degree $0$.

\paragraph{The correct (good) truncation that \emph{is} a cdga.}

\begin{definition}[Good connective truncation]\label{def:good-trunc}
For $n\in\mathbb{Z}$ define the \emph{good lower truncation} $\tau_{\ge n}^{\mathrm{dg}}C^\ast$ by
\[
\bigl(\tau_{\ge n}^{\mathrm{dg}}C^\ast\bigr)^k :=
\begin{cases}
0 & k<n,\\
Z^n:=\ker(d:C^n\to C^{n+1}) & k=n,\\
C^k & k>n,
\end{cases}
\]
with differential the restriction of $d$ (i.e.\ $Z^n\hookrightarrow C^{n+1}$ via $d$) and
multiplication induced from $C^\ast$ by restriction. The unit is the same as in $C^\ast$.
\end{definition}

\begin{proposition}\label{prop:good-trunc-is-cdga}
If $C^\ast$ is a cdga, then $\tau_{\ge n}^{\mathrm{dg}}C^\ast$ is a cdga and the canonical map
\[
\pi_{\ge n}: C^\ast \longrightarrow \tau_{\ge n}^{\mathrm{dg}}C^\ast
\]
is a cdga morphism. Moreover, $\pi_{\ge n}$ induces isomorphisms
$H^k(C^\ast)\cong H^k(\tau_{\ge n}^{\mathrm{dg}}C^\ast)$ for all $k>n$.
\end{proposition}

\begin{proof}
Closure under multiplication: degrees $>n$ are unchanged; for degree $n$ we use
$Z^n\subset C^n$, and if $x\in Z^n$, $y\in Z^m$ then
$d(xy)=d(x)\,y+(-1)^n x\,d(y)=0$, so $xy\in Z^{n+m}$; if $y\in C^m$ with $m>n$,
then $xy\in C^{n+m}$ which belongs to $\tau_{\ge n}^{\mathrm{dg}}C^{n+m}$.
Leibniz and graded commutativity are inherited from $C^\ast$.
The cohomology claim is the standard property of good truncation (as in
Remark~\ref{rmk:trunc-cohom-iso}, with $Z^n$ replacing $C^n/B^n$ at degree $n$).
\end{proof}

\begin{remark}[Why the good truncation works; why the na\"\i ve one fails]
The obstacle in $t_{[n}$ is the quotient $C^n/B^n$ at degree $n$:
$B^\ast$ is \emph{not} a multiplicative ideal in a general cdga, so multiplication by classes
$[a]\in C^n/B^n$ is not well-defined. Replacing $C^n/B^n$ by the submodule $Z^n$ avoids
quotients and keeps multiplicativity. Dually, no honest upper truncation can be a sub/cokernel
cdga functor in general because degrees \(>n\) interact multiplicatively with lower degrees.
\end{remark}

\paragraph{Some positive special cases.}

\begin{proposition}[When upper truncation behaves]\label{prop:upper-special}
Suppose $C^\ast$ is \emph{connective} ($C^{<0}=0$) and \emph{square-zero above $n$}, i.e.
$C^{>n}\cdot C^{\ge 0}=0$. Then $t_{n]}C^\ast$ inherits a cdga structure (as a quotient cdga)
and $\iota:t_{n]}C^\ast\to C^\ast$ is a cdga morphism.
\end{proposition}

\begin{proof}
Under the hypothesis $C^{>n}\cdot C^{\ge 0}=0$, products never leave the range $\le n$,
so cutting off degrees \(>n\) is closed under multiplication. Since $\ker d^n\subset C^n$ is
a subring and $d$ is still a derivation, the usual quotient-by-degrees construction defines a cdga.
\end{proof}

\begin{example}[Trivial case]\label{ex:deg0}
If $C^\ast$ is concentrated in degree $0$ (an ordinary commutative ring),
then all the truncations coincide with $C^\ast$ and are cdga's.
This is the only truly general case where both $t_{[n}$ and $t_{n]}$ behave without
extra hypotheses (cf.\ Proposition~\ref{prop:naive-fails}).
\end{example}

\paragraph{Homotopical truncations (Postnikov) and cdga’s.}
In derived settings one often wants Postnikov truncations \(\tau_{\ge n},\tau_{\le n}\)
in the \emph{derived} category. There, truncations exist functorially and preserve
multiplicative structure \emph{up to coherent homotopy}. Concretely, for cdga’s in arbitrary
characteristic, one passes to \(E_\infty\)–algebras and performs truncations in that model,
then (in characteristic \(0\)) rectifies back to cdga’s. At the strict (underived) level, the
only unconditional, strictly multiplicative truncation is the good connective one
from Definition~\ref{def:good-trunc}.

\subsection{Truncations of cdga's at degree \texorpdfstring{$0$}{0}}\label{subsec:trunc-0}

Let $(C^\ast,d,\cdot)$ be a commutative differential graded $A$--algebra (cdga), cohomological
grading ($d:C^k\to C^{k+1}$). Recall the (complex-theoretic) truncations
\[
\bigl(t_{n]}C^\ast\bigr)^k=
\begin{cases}
C^k & k<n,\\
Z^n:=\ker(d:C^n\to C^{n+1}) & k=n,\\
0 & k>n,
\end{cases}
\qquad
\bigl(t_{[n}C^\ast\bigr)^k=
\begin{cases}
0 & k<n,\\
C^n/B^n & k=n,\\
C^k & k>n,
\end{cases}
\]
with the evident differentials ($B^n=d(C^{n-1})$). We write $t_{[0}t_{0]}C^\ast$ for the composite.

\begin{remark}[Leibniz failure for $t_{n]}$ when $n>0$]\label{rmk:leb-fails-n>0}
Assume $n>0$ and take $a\in \bigl(t_{n]}C^\ast\bigr)^r$, $b\in \bigl(t_{n]}C^\ast\bigr)^s$ with $r+s=n$
and (say) $r\neq n$ (so $s<n$). In $C^\ast$ we have
\[
d(a\cdot b)\;=\;d(a)\cdot b+(-1)^r\,a\cdot d(b),
\]
and at least one of $d(a),d(b)$ need not vanish (because $r<n$ or $s<n$).
But $\bigl(t_{n]}C^\ast\bigr)^{n+1}=0$, hence $d(a\cdot b)$ must be $0$ in $t_{n]}C^\ast$.
Unless $d(a)\cdot b$ and $a\cdot d(b)$ both vanish \emph{in $C^\ast$}, the Leibniz rule is violated
when viewed in $t_{n]}C^\ast$. Therefore $t_{n]}$ cannot in general carry a cdga structure compatible
with the one on $C^\ast$ unless $n=0$.
\end{remark}

The only truncations that \emph{always} preserve a cdga structure are the ones at degree \(0\):

\begin{equation}\label{eq:maps-1.7-1.8}
t_{0]}C^\ast\;\longrightarrow\; C^\ast,
\qquad
t_{0]}C^\ast\;\longrightarrow\; t_{[0}\,t_{0]}C^\ast.
\end{equation}

\begin{proposition}\label{prop:t0-and-composite-are-cdga}
Let $C^\ast$ be a cdga. Then $t_{0]}C^\ast$ and $t_{[0}t_{0]}C^\ast$ are cdga's. Moreover,
the maps in \eqref{eq:maps-1.7-1.8} are morphisms of cdga's.
\end{proposition}

\begin{proof}
\textbf{(A) The cdga $t_{0]}C^\ast$.}
As a complex,
\[
\bigl(t_{0]}C^\ast\bigr)^k=
\begin{cases}
C^k & k<0,\\
Z^0:=\ker(d:C^0\to C^1) & k=0,\\
0 & k>0,
\end{cases}
\qquad d\ \text{the restriction of }d \text{ on }k<0.
\]
Define the multiplication $\cdot$ on $t_{0]}C^\ast$ by
\[
a\cdot b\ :=
\begin{cases}
ab & \deg a,\deg b\le 0,\\
0 & \text{otherwise},
\end{cases}
\]
where $ab$ is the product in $C^\ast$.

\emph{Well-definedness in degree $0$.}
If $a,b\in Z^0$, then
$d(ab)=d(a)b+a\,d(b)=0$, so $ab\in Z^0$. For other degrees, the rule is tautological.

\emph{Associativity/commutativity.}
Inherited from $C^\ast$ on the only nontrivial case (both degrees $\le 0$); the zero
cases are automatic.

\emph{Leibniz rule.}
Let $\delta$ be the differential on $t_{0]}C^\ast$ (restriction of $d$). The only nontrivial
checks are when at least one factor has degree $0$.

If $\deg a=0$, $\deg b<0$, then $\delta(a\cdot b)=d(ab)=d(a)b+a\,d(b)=a\,d(b)=a\cdot \delta(b)$,
while $\delta(a)\cdot b=d(a)\,b=0$. Hence $\delta(a\cdot b)=\delta(a)\cdot b+(-1)^{\deg a}a\cdot \delta(b)$.

If $\deg a<0$, $\deg b=0$, then $\delta(a\cdot b)=d(ab)=d(a)b+(-1)^{\deg a}a\,d(b)=d(a)b$,
and $\delta(a)\cdot b=d(a)b$, $a\cdot\delta(b)=a\,d(b)=0$. Again Leibniz holds.
If both degrees are $0$ the three terms vanish. Therefore $t_{0]}C^\ast$ is a cdga.

\emph{The map $t_{0]}C^\ast\to C^\ast$.}
This is the inclusion on $k<0$ and on $Z^0\hookrightarrow C^0$, and zero above; it clearly
preserves multiplication and differentials, hence is a cdga morphism.

\smallskip
\textbf{(B) The cdga $t_{[0}t_{0]}C^\ast$.}
As a complex,
\[
\bigl(t_{[0}t_{0]}C^\ast\bigr)^k=
\begin{cases}
0 & k<0,\\
Z^0/B^{-1} & k=0\quad (B^{-1}:=d(C^{-1})),\\
C^k & k>0,
\end{cases}
\]
with $d$ induced from $C^\ast$. Define the product $\star$ by
\[
(a+ B^{-1})\ \star\ (b+B^{-1})\ :=
\begin{cases}
ab + B^{-1} & \deg a=\deg b=0,\\
0 & \text{otherwise}.
\end{cases}
\]
\emph{Well-definedness in degree $0$.}
If we change representatives $a\mapsto a+d(a')$, $b\mapsto b+d(b')$ with $a',b'\in C^{-1}$, then
\[
(a+d a')(b+d b')=ab+d(a'b)+a\,d(b')+d(a')\,d(b').
\]
Now $d(a'b)\in B^{-1}$, and $a\,d(b')=d(ab')-d(a)\,b'\in B^{-1}$ because $a\in Z^0$.
Also $d(a')\,d(b')=d(a'\,d(b'))\in B^{-1}$. Hence the class of $ab$ in $Z^0/B^{-1}$ is
independent of representatives.

\emph{Associativity/commutativity.}
Immediate in degree $0$ from $C^\ast$; zero elsewhere, hence automatic.

\emph{Leibniz rule.}
Trivial: if any factor has $\deg\neq 0$, every term is zero; for degree $0$, all three
terms are in degree $0$ and $d$ vanishes on $Z^0/B^{-1}$.

\emph{The map $t_{0]}C^\ast\to t_{[0}t_{0]}C^\ast$.}
This is the projection $Z^0\twoheadrightarrow Z^0/B^{-1}$ in degree $0$ and zero elsewhere;
it respects the multiplications by the definitions above, so it is a cdga morphism.
\end{proof}

\begin{corollary}\label{cor:qi-acyclic-except-0}
If $C^\ast$ is a cdga with $H^k(C^\ast)=0$ for $k\neq 0$, then both maps
\[
t_{0]}C^\ast\longrightarrow C^\ast,
\qquad
t_{0]}C^\ast\longrightarrow t_{[0}t_{0]}C^\ast
\]
are quasi-isomorphisms of cdga's.
\end{corollary}

\begin{proof}
On cohomology, $t_{0]}C^\ast\to C^\ast$ induces isomorphisms in degrees $<0$ (identity)
and in degree $0$ (since $H^0(C^\ast)\cong Z^0/B^{-1}$ and $H^0(t_{0]}C^\ast)=Z^0$ maps onto it
with kernel $B^{-1}$), and there is no cohomology above $0$. Similarly, the projection
$t_{0]}C^\ast\to t_{[0}t_{0]}C^\ast$ kills exactly $B^{-1}$ in degree $0$ and is an isomorphism on
the only surviving cohomology group; elsewhere both sides have zero cohomology. Hence both are quasi-isomorphisms.
\end{proof}

\begin{remark}[On replacing a cdga by a ring in degree zero]\label{rmk:replace-by-H0}
Under the hypothesis of Corollary~\ref{cor:qi-acyclic-except-0} we deduce that $C^\ast$
is quasi-isomorphic (as a cdga) to the ordinary ring $H^0(C^\ast)$ concentrated in degree $0$.
This is frequently exploited in computations: one may treat a ring as a cdga (and conversely)
whenever cohomology is concentrated in degree $0$.

Two caveats:
(i) the maps $t_{0]}\to C^\ast$ and $t_{0]}\to t_{[0}t_{0]}$ point in different directions, so neither
alone is the quasi-isomorphism $C^\ast\to H^0(C^\ast)$; one passes to the homotopy category
(localizing at quasi-isomorphisms) to invert them.
(ii) Strict categorical forgetful functors to complexes can lose information after localization.
These issues mirror the monoidal non-compatibility discussed for Dold--Kan; a robust resolution
is to work with $E_\infty$–algebras, where commutativity is preserved \emph{up to coherent homotopy}
and such truncation/rectification procedures are homotopically meaningful.
\end{remark}

\subsection{\texorpdfstring{$E_\infty$}{E-infinity}-algebras}\label{subsec:Einfty}

Informally, an $E_\infty$–algebra over a commutative ring $R$ is a
``commutative dg–algebra up to coherent homotopies'': the multiplication is
associative and commutative only up to homotopies, which themselves satisfy
higher coherence conditions.

\begin{definition}[$E_\infty$–algebra (operadic viewpoint)]
Fix a cofibrant $E_\infty$–operad $\mathcal{E}$ in the symmetric monoidal
category of chain complexes $\Ch(R)$ (with the usual tensor and Koszul signs).
An \emph{$E_\infty$–algebra over $R$} is an algebra over $\mathcal{E}$, i.e.\ a
chain complex $A\in\Ch(R)$ equipped with structure maps
\[
\mathcal{E}(n)\otimes A^{\otimes n}\longrightarrow A\qquad(n\ge 0)
\]
compatible with the operad compositions, units and symmetric actions.
A morphism of $E_\infty$–algebras is a morphism of $\mathcal{E}$–algebras.
\end{definition}

\begin{remark}[Homotopical purpose]
Choosing $\mathcal{E}$ cofibrant makes the category $\EAlg_R$ homotopically
well-behaved (model structure transferred from $\Ch(R)$): weak equivalences and
fibrations are created by the forgetful functor to $\Ch(R)$. Thus
``equivalence of $E_\infty$–algebras'' means quasi-isomorphism on underlying
complexes.
\end{remark}

\subsubsection*{A warm-up: \texorpdfstring{$A_\infty$}{A-infinity}–algebras}

\begin{definition}[$A_\infty$–algebra (Stasheff)]
An \emph{$A_\infty$–algebra} over $R$ is a graded $R$–module $A$ with degree
$2-n$ operations
\[
m_n:A^{\otimes n}\longrightarrow A\qquad(n\ge 1)
\]
satisfying the Stasheff identities
\[
\sum_{r+s+t=n}(-1)^{r+st}\,
m_{r+1+t}\bigl(\Id^{\otimes r}\otimes m_s\otimes \Id^{\otimes t}\bigr)=0
\qquad(n\ge 1).
\]
Here $m_1$ is a differential ($m_1^2=0$), $m_2$ is associative up to the
homotopy $m_3$, the homotopy $m_3$ is coherent up to $m_4$, and so on.
A strict dga is an $A_\infty$–algebra with $m_n=0$ for $n\ge 3$.
\end{definition}

\begin{example}[Classical sources]
Singular cochains $C^\ast(X;R)$ carry natural $A_\infty$ (indeed $E_\infty$)
structures; the higher $m_n$ encode the classical cup-$i$ operations
and their compatibilities (Steenrod operations in mod-$p$ homology).
\end{example}

\subsubsection*{From cdga’s to \texorpdfstring{$E_\infty$}{E-infinity}: rectification vs.\ coherence}

Two operads play a central role: the \emph{commutative} operad $\Com$ and a
chosen cofibrant $E_\infty$–operad $\mathcal{E}$ equipped with a weak
equivalence of operads $\mathcal{E}\to\Com$.

\begin{remark}\label{fact:cdga-to-Einfty}
Any commutative differential graded $R$–algebra (cdga) admits a canonical
$E_\infty$–structure: viewing a cdga as a $\Com$–algebra and pulling back
along $\mathcal{E}\to\Com$ yields an $\mathcal{E}$–algebra on the same
underlying complex.
\end{remark}

\begin{proof}[Explanation]
A cdga is, by definition, an algebra over $\Com$. Composition with the operad
map $\mathcal{E}\to\Com$ gives an $\mathcal{E}$–algebra structure; underlying
differential and multiplication do not change, only the higher coherences are
now governed by $\mathcal{E}$ (trivial for strict cdga’s).
\end{proof}

\begin{remark}\label{fact:morphism}
A morphism of cdga’s is (by the same pullback) a morphism of the corresponding
$E_\infty$–algebras.
\end{remark}

\begin{proof}[Explanation]
Algebra maps over $\Com$ compose with $\mathcal{E}\to\Com$ to give
$\mathcal{E}$–algebra maps. Functoriality is immediate.
\end{proof}

\begin{remark}\label{fact:qi}
A quasi-isomorphism of cdga’s is a weak equivalence of the corresponding
$E_\infty$–algebras. In particular, it becomes invertible in the homotopy
category $\Ho(\EAlg_R)$ (admits a homotopy inverse).
\end{remark}

\begin{proof}[Explanation]
The forgetful functor $\EAlg_R\to\Ch(R)$ creates weak equivalences; hence an
underlying quasi-isomorphism is a weak equivalence in $\EAlg_R$. Passing to the
homotopy category inverts such maps.
\end{proof}

\begin{remark}[Characteristic issues and rectification]
Over a characteristic $0$ ring (e.g.\ $\mathbb{Q}$), there is a Quillen
equivalence between $E_\infty$–algebras and cdga’s (``rectification''):
every $E_\infty$–algebra is weakly equivalent to a strict cdga. In positive
characteristic this fails in general; divided power phenomena (and
Dyer–Lashof/Steenrod operations) obstruct strict commutativity, which is
precisely why $E_\infty$–algebras are the correct replacement for cdga’s.
\end{remark}

\subsubsection*{Basic properties and examples}

\begin{example}[CoChains are $E_\infty$]
For any simplicial set (or CW complex) $X$, the cochain complex $C^\ast(X;R)$
admits a natural $E_\infty$–structure refining the cup product. The induced
product on cohomology is the usual graded-commutative cup product.
\end{example}

\begin{example}[Strict cdga’s as $E_\infty$ with trivial higher operations]
If $A$ is a cdga, its image in $\EAlg_R$ has all ``higher'' operations
strictly determined by the ordinary multiplication—no extra homotopies are
needed. Thus cdga’s embed fully faithfully into $E_\infty$–algebras via
Fact~\ref{fact:cdga-to-Einfty}.
\end{example}

\begin{remark}[Why $E_\infty$ solves §\ref{subsec:simplicial-to-dga} issues]
Dold–Kan does not preserve the \emph{symmetric monoidal structure} strictly.
Replacing strict commutativity by $E_\infty$–commutativity yields functors and
equivalences that are homotopically natural (e.g.\ normalized chains of a
simplicial commutative ring form an $E_\infty$–algebra canonically; in
characteristic $0$ this rectifies to a cdga).
\end{remark}

\subsubsection*{Optional: $A_\infty$ morphisms (for intuition)}
A morphism $f:A\to B$ of $A_\infty$–algebras is a collection of maps
$f_n:A^{\otimes n}\to B$ of degree $1-n$ satisfying identities paralleling the
Stasheff relations (compatibility with $m_\bullet$). A strict dga map is the
special case $f_1$ a chain map and $f_n=0$ for $n\ge 2$.
This mirrors the $E_\infty$ situation where higher coherences also decorate
morphisms in a homotopically meaningful way.

\medskip

Summarizing the items requested in Facts~1.2.30:
\begin{enumerate}
\item Every cdga is canonically an $E_\infty$–algebra (pullback along
$\mathcal{E}\to\Com$).
\item Every cdga morphism is a morphism of $E_\infty$–algebras.
\item Every quasi-isomorphism of cdga’s is a weak equivalence in $\EAlg_R$; hence
an isomorphism in $\Ho(\EAlg_R)$ and admits a homotopy inverse there.
\end{enumerate}
These are the precise formal underpinnings of the ``black box'' you flagged at the
end of \S\ref{subsec:trunc-0}: $E_\infty$–algebras provide the homotopically correct
framework where cdga’s and rings can be compared and replaced up to equivalence.

